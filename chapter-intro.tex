% -*- coding: utf-8; -*-

\chapter{Introdução}

A restauração estrutural de modelos geológicos permite quantificar a intensidade da deformação em unidades geológicas, analisar estruturas complexas em formação, além de ponderar sobre a função de uma estrutura para acumulação de um bem mineral~\cite{GarciaTese}.

Dentre os método usados na restauração estrutural, um dos mais abordados é o balanceamento de seções geológicas. No balanceamento de seções são usados cortes transversais resultantes da interseção de planos verticais com modelos geológicos tridimensionais. As seções geológicas definidas pelo geólogo são manipuladas segundo uma série de técnicas, por sua vez baseadas em premissas geológicas com o objetivo de se obter o estado original, indeformado. A modelagem de seções geológicas no sentido contrário ao tempo é denominado \emph{restauração}~\cite{Fossen}. Sob o ponto de vista computacional, a restauração de seções geológicas apresenta ferramentas 2D simples, rápidas e eficientes~\cite{GarciaTese}.

A aplicação de técnicas de restauração de seções geológicas possibilita inferir uma série de informações como: a quantidade de encurtamento ou extensão uma região, análises sobre a evolução estrutural da área, definição de épocas de atividades das falhas, de movimentação de sal e obter uma estimativa do momento de desenvolvimento de uma armadilha de hidrocarbonetos~\cite{DURANDRIARD-3D, Guedes}. No geral, no método de restauração estrutural as seções transversais são criadas paralelamente à direção principal de transporte tectônico. Entretanto, em modelos com falhas mais complexas ou com transporte em direções diversas, algumas das premissas geológicas adotadas podem não ser atendidas~\cite{GarciaTese}. Em função disso caso não seja bem comportada a movimentação tectônica da região em estudo, será necessário adotar uma modelagem tridimensional para o problema. Portanto, o uso apenas de seções geológicas para restauração estrutural pode não apresentar o que ocorre efetivamente no volume entre seções.

Para se chegar a uma interpretação estrutural mais completa da restauração de um modelo geológico pode-se partir para uma restauração 3D. Nesse caso, todo o modelo geológico precisa passar pelo processo de restauração de maneira completa. Isso envolve acrescentar uma série de restrições, etapas de verificação da consistência do modelo e uma representação discreta mais complexa, como o uso de malhas de elementos finitos~\cite{DURANDRIARD2010441}. Esse tipo de solução requer a definição de modelos geológicos mais complexos, que não estão completamente difundidos na área. Além disso, a restauração 3D requer mais informações e pode envolver um alto custo computacional para se chegar em um bom resultado.

\section{Motivação}

A restauração de seções geológicas tem se mostrado ao longo do tempo um método eficiente de auxílio na interpretação geológico-estrutural. As seções geológicas podem ser vistas como amostras discretas de modelos 3D. Ao aumentar o número de seções no processo de restauração, as informações geradas sobre o transporte tectônico ficam mais ricas em relação ao modelo completo. Essas informações associadas a cada seção geológica integradas, podem ser usadas para se fazer uma extrapolação dos eventos tectônicos que ocorrem no volume entre as seções do modelo.

O modelo geológico de uma determinada região é formado por diversos tipos de estruturas e com uma metodologia própria de criação~\cite{Rodrigues}. Em modelos voltados à restauração de seções podem ser usados aqueles com superfícies tridimensionais representado o topo das camadas geológicas ali presentes. As seções geológicas terão linhas representando as superfícies interceptadas pelo plano transversal das seções.

Ao realizar a restauração das seções geológicas, as linhas que representam as superfícies geológicas sofrem transformações gerando campos de deslocamento e de deformações de acordo com a técnica de restauração adotada. No entanto, essas transformações não são aplicadas às superfícies o que gera uma dissociação entre seções e superfícies. As linhas das seções, contudo, possuem a informação do quanto se deformaram entre um passo e outro da restauração das seções.

Essas informações de deformação da seção geológica podem ser usadas para mapear o comportamento da superfície geológica e assim obter uma superfície coerente com as transformações sofridas pelas seções.

O mapeamento de informações para realizar a caracterização de uma ou mais estruturas dentro da Geologia é uma ferramenta bastante presente. Cita-se como casos de uso o mapeamento geológico-estrutural que utiliza técnicas para descrever estruturas geológicas e suas distribuições no espaço de uma determinada região~\cite{Borges, Felipe}, e o mapeamento de paleorrelevos. Esse último é interpretado como sendo o estado da superfície geológica em um dado momento no tempo~\cite{Archela}.

Adicionalmente, as informações de movimentação tectônica produzida pelas seções também podem ser estendidas para um mapeamento do volume. Nesse caso, um ponto qualquer no domínio do modelo geológico pode vir a sofrer um deslocamento com base na movimentação das seções geológicas adjacentes.

\section{Objetivos}

Com o intuito de obter a representação de superfícies, não apenas na época de sua sedimentação mas em cada etapa onde ocorreu movimentação tectônica de modelos geológicos submetidos a um regime distensivo, este trabalho tem por objetivo realizar o mapeamento de superfícies geológicas com base em restauração de seções geológicas. Atingido esse objetivo, mapear também a movimentação do volume geológico pela combinação dos dados de seções e superfícies.

Isso permite realizar um acompanhamento cronológico mais assertivo das superfícies geológicas, com a possibilidade de, dado uma configuração corrente e um tempo geológico, obter a nova configuração neste novo tempo. A mesma ideia pode ser utilizada para a movimentação do volume.

\section{Metodologia}

O mapeamento das superfícies pode ser construído a partir da superfície de interesse em sua configuração atual representada por uma malha de triângulos e o uso de dados das seções como pontos de controle que irão definir a direção da movimentação. De outra forma, os pontos de controle servem como parâmetros para realizar uma deformação na malha da superfície.

Para o volume geológico, a metodologia parte da definição de uma nuvem de pontos no domínio tridimensional que represente a geometria do volume, para tal, cada ponto precisa ter um atributo referente à camada geológica que o contém. As informações de movimentação das seções e das superfícies (obtidas anteriormente) também estão discretizadas em pontos neste mesmo espaço e são usadas para guiar a movimentação dos pontos vizinhos, gerando um deslocamento em cadeia por todo o volume.

As superfícies geológicas sofrem uma deformação com base no deslocamento apresentado pelas linhas presentes nas seções geológicas. A deformação da superfície neste trabalho é feita segundo a ideia de suavização de superfícies apresentada por Botsch \emph{et al.}~\cite{Botsch}. Em especial, a suavização que busca obter superfícies do tipo \emph{fairing} onde se produz formas suavizadas ao máximo.

Isso é alcançado a partir da minimização de uma energia. A depender da ordem dessa energia, um tipo de superfície suavizada é obtida. Em relação à energia de superfície membrana, que é do 2ª ordem, ela minimiza a área da superfície. A energia de placas finas é descrita com uma formulação do 4ª ordem e ao minimizá-la se chega a uma \emph{superfície de mínima curvatura}. A suavização de superfície utilizada neste trabalho busca obter uma \emph{superfície de mínima variação} e é baseada em uma formulação de 6ª ordem~\cite{Botsch}.

A abordagem que busca uma superfície de mínima curvatura (4ª ordem) é base do método de interpolação suavizada discreta (\emph{discrete smooth interpolation})~\cite{DSI} usada na modelagem geométrica de superfícies geológicas complexas do sistema GOCAD~\cite{GOCAD}.

A implementação computacional deste trabalho foi realizado no Sistema Recon MS~\cite{ReconTecgraf}, utilizado diretamente na indústria de óleo e gás, mais especificamente na PETROBRAS no setor de extração, fornecendo ferramentas para restauração de seções geológicas, entre outros recursos relacionados.

\section{Escopo}

O presente trabalho encontra-se dividido em 6 capítulos, resumidos a seguir.

No capítulo 2 é apresentado o processo de restauração de seções geológicas no ambiente computacional usado como base de desenvolvimento deste trabalho, bem como tudo que envolve essa atividade para obtenção da movimentação tectônica.

O capítulo 3 introduz o mapeamento usado nas seções geológicas dentro do Sistema Recon. Incluindo a apresentação do mapeamento das entidades geológicas durante a restauração de seção.

Os capítulos 4 e 5 mostram a forma de preparação dos dados a partir das seções geológicas para o mapeamento de superfícies e volume, respectivamente. Em cada capítulo são apresentados a metodologia de implementação destes mapeamentos, casos de uso e discussão dos resultados.

O capítulo 6 possui a conclusão, sugestão de trabalhos posteriores e as últimas considerações acerca deste estudo.
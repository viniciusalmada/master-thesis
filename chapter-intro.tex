% -*- coding: utf-8; -*-

\chapter{Introdução}

\section{Motivação}

% Eu quero chegar no problema do mapeamento de paleotopografia
A restauração de uma seção geológica ao seu estado original, anterior à deformação, é uma importante parte da interpretação estrutural.\cite{Fossen} Essa interpretação costuma levar em conta diversos parâmetros que podem partir desde o mapeamento fornecido pelo uso de canhões de ar na obtenção de dados sísmicos até a modelagem digital das seções transversais planas com os dados das camadas geológicas e falhas encontradas. Essa maneira bidimensional de trabalhar modelos geológicos é a que levanta resultados mais rapidamente e envolve métodos e premissas mais simples do que a forma, talvez, mais intuitiva que seria com a utilização direta do volume, uma vez que naturalmente os processos geológicos ocorrem no espaço tridimensional.

Apesar disso, os resultados da restauração e balanceamento de seções geológicas são bons o suficiente para validar a interpretação geológica, analisar a evolução estrutural da área, definir épocas de atividades das falhas, de movimentação de sal e ainda auxiliar na exploração de sistemas petrolíferos.\cite{Guedes} Pela própria definição de seção transversal percebe-se que se trata de uma discretização do volume, logo, quanto mais seções um modelo tiver, mais completos e claros todos esses resultados serão obtidos. A depender da forma que ocorreu a movimentação, esta restauração também pode ocorrer em estágios ou, uma camada de cada vez e assim realizar avaliações parciais.

Quando se trabalha com um conjunto de seções geológicas pertencentes a um mesmo modelo pode-se fazer uso de marcos ou etapas de restauração referentes àqueles estágios em todas as seções envolvidas, isto pode servir para manter uma certa coerência com o processo natural de movimentação tectônica (que ocorre de forma simultânea) e também permite extrair informações e interpretações parciais durante a atividade de restauração. Como por exemplo, a superfície de paleorelevos que, neste contexto, pode ser definida como a superfície de uma determinada camada de uma certa idade, associada a uma etapa da restauração, em outras palavras: ao finalizar a restauração de uma camada geológica, cria-se uma superfície por interpolação dos dados referentes àquela camada em cada uma das seções envolvidas. Esta superfície vem a ser a representação no espaço do topo daquela camada na época de sua sedimentação.

















\section{Objetivos}

\section{Escopo}



% -*- coding: utf-8; -*-

\chapter{Introdução}

\section{Motivação}

\iffalse
A restauração de uma seção geológica ao seu estado original, anterior à deformação, é uma importante parte da interpretação estrutural.\cite{Fossen} Essa interpretação costuma levar em conta diversos parâmetros que podem partir desde o mapeamento fornecido pelo uso de canhões de ar na obtenção de dados sísmicos até a modelagem digital das seções transversais planas com os dados das camadas geológicas e falhas encontradas.

Essa maneira bidimensional de trabalhar modelos geológicos é a que levanta resultados mais rapidamente e envolve métodos e premissas mais simples do que a forma, talvez, mais intuitiva que seria com a utilização direta do volume, uma vez que naturalmente os processos geológicos ocorrem no espaço tridimensional.

Apesar disso, os resultados da restauração e balanceamento de seções geológicas são bons o suficiente para validar a interpretação geológica, analisar a evolução estrutural da área, definir épocas de atividades das falhas, de movimentação de sal e ainda auxiliar na exploração de sistemas petrolíferos.\cite{Guedes} Pela própria definição de seção transversal percebe-se que se trata de uma discretização do volume, logo, quanto mais seções um modelo tiver, mais completos e claros todos esses resultados serão obtidos. A depender da forma que ocorreu a movimentação tectônica, esta restauração também pode ocorrer em estágios ou, uma camada de cada vez.

Quando se trabalha com um conjunto de seções geológicas de um mesmo modelo pode-se fazer uso de marcos ou etapas de restauração referentes àqueles estágios em todas as seções envolvidas, isto pode servir para manter uma coerência com o processo natural de movimentação tectônica (que ocorre de forma simultânea em todas seções) e também permite extrair informações e interpretações parciais durante a atividade de restauração. Como por exemplo, a superfície de paleo-relevos que, neste contexto, pode ser definida como a superfície de uma determinada camada em um certo período da história, associada a uma etapa da restauração, em outras palavras: ao finalizar a restauração de uma camada geológica, cria-se uma superfície por interpolação dos dados referentes àquela camada em cada uma das seções envolvidas. Esta superfície vem a ser a representação no espaço do topo daquela camada na época de sua sedimentação.

A obtenção de superfícies geológicas a partir de seções é uma maneira bastante eficaz de se ter um modelo quase tridimensional, ou 2,5D e com isso ter informações mais completas como um todo. No entanto, essa forma apresentada necessita que a camada do topo em todas as seções envolvidas esteja restaurada e para se chegar neste estágio, há outras etapas intermediárias. Em cada uma dessas etapas ocorre algum tipo de movimentação tectônica localmente nas seções que acaba não sendo levada em conta nas superfícies.

Uma modelagem geológica baseada em restauração de seções precisa estar alinhada com a movimentação tectônica gerada por elas. Se houver inclusão de superfícies e suas próprias metodologias de restauração, pode haver uma grande incompatibilidade na forma de se fazer tal processo.

Uma abordagem para tal problema é a utilização de mapeamento de informações das seções para as superfícies com uso de alguma interface de dados que possa ser manipulada através de um procedimento numérico de acordo com a movimentação produzida pelas seções. Em outras palavras, a movimentação tectônica resultante de cada etapa da restauração de seções seria usada como base para mapear o mesmo comportamento nas superfícies e produzir paleo-relevos de forma análoga à aplicação de uma deformação em uma malha no espaço.

O mapeamento já é uma ferramenta bastante usada na Geologia e de várias formas. Na obtenção de dados sísmicos, como já citado, pode ser usado um canhão de ar em direção ao solo ou rochas e geofones (em terra) ou hidrofones (no mar) captam a reflexão das ondas sonoras e assim assim chega-se à chamada sísmica da região através do mapeamento do que é captado.\cite{Fossen} Um outro caso, talvez mais geral é o mapeamento geológico-estrutural que, sucintamente, faz uso de técnicas para descrever estruturas geológicas e suas distribuições no espaço de uma determinada região e assim ter uma base de análise.\cite{Borges, Felipe} Já o mapeamento de paleo-relevos, no contexto deste trabalho, busca conseguir uma representação da superfície de uma camada geológica levando em conta a movimentação tectônica produzida pela restauração de seções geológicas.

Este mapeamento é capaz de fornecer dados de movimentação da superfície e com a combinação do já conhecido movimento das seções, é possível mapear a movimentação do volume geológico, obtendo assim uma movimentação tridimensional inteiramente baseada na restauração de seções que, como já dito, possui um fluxo de trabalho mais rápido.

\fi

\section{Objetivos e Metodologia}

Com o intuito de obter a representação de superfícies, não apenas na época de sua sedimentação mas em cada etapa onde ocorreu movimentação tectônica em modelos geológicos de regime extensional, este trabalho tem por objetivo realizar o mapeamento de paleo-relevos com base em restauração de seções. E a partir disso, mapear também a movimentação do volume pela combinação dos dados de seções e superfícies.

Isto permite realizar um acompanhamento cronológico mais assertivo das superfícies geológicas, com a possibilidade de, dado uma configuração corrente e um tempo geológico, obter a nova configuração neste novo tempo. A mesma metodologia pode ser utilizada para a movimentação do volume.

O mapeamento dos paleo-relevos pode ser construído a partir da superfície de interesse em sua configuração atual representada por uma malha de triângulos e o uso de dados das seções como pontos de controle que irão definir a direção da movimentação, ou mais claramente, os pontos de controle servirão como parâmetros para realizar uma deformação na malha da superfície.

Agora para o volume, a metodologia parte da definição de uma nuvem de pontos no domínio tridimensional que represente a geometria do volume, onde cada ponto precisa ter um atributo referente à camada geológica que o contém. As informações de movimentação das seções e das superfícies (obtidas anteriormente) também estão discretizadas em pontos neste mesmo espaço e são usadas para guiar a movimentação dos pontos vizinhos, gerando um deslocamento em cadeia por todo o volume. 

\section{Escopo}

O presente trabalho encontra-se dividido em 4 capítulos listados adiante.

No capítulo 2 é apresentado o processo de restauração de seções geológicas no ambiente computacional usado como base de desenvolvimento deste trabalho, bem como tudo que envolve essa atividade para obtenção da movimentação tectônica.

Já o capítulo 3 mostra a forma de preparação dos dados das seções geológicas para o mapeamento de superfícies e volume, a metodologia de implementação destes mapeamentos, casos de uso e discussão dos resultados.

O capítulo 4 possui a conclusão, sugestão de trabalhos posteriores e as últimas considerações acerca deste estudo.





% -*- coding: utf-8; -*-

\chapter{Conclusão}

Este trabalho propôs um mapeamento do comportamento de superfícies e volume baseado na restauração de seções geológicas. Uma maneira de levantar informações a respeito do comportamento tridimensional a partir da movimentação tectônica gerada pelo balanceamento das seções geológicas.

Inicialmente foi apresentado uma breve caracterização do que se trata a restauração de seções geológicas com enfoque maior naquilo que o Sistema Recon MS, base de desenvolvimento, oferece para realizar tal atividade. Dentre os recursos fornecidos pode-se citar a estrutura de dados topológicos HED, geração de malhas em seções, transformações geométricas e por fim, as linhas de mapeamento.

Este mapeamento realizado nas seções foi base para a criação das \emph{LMModels} e com isso se obteve o mapeamento das entidades geológicas ao longo da restauração. Ao exportar as \emph{LMModels} para um ambiente multisseções, foi possível associar este mapeamento às superfícies tridimensionais. A organização e uso desses dados de \emph{LMModels} serviram como parâmetros para, através de um método numérico, realizar uma deformação às superfícies a fim de submetê-las à mesma movimentação das seções. O responsável por essa deformação vem do processo de minimização de uma energia de alta ordem, tri-harmônica, que resulta em uma superfície de mínima variação.

Em relação ao volume, foi idealizado que ele fosse discretizado em uma nuvem de pontos contida no domínio do modelo. Os pontos das malhas das seções também 


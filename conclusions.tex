% -*- coding: utf-8; -*-

\chapter{Conclusão}

\section{Resumo}

Este trabalho propôs um mapeamento do comportamento de superfícies e volume baseado na restauração de seções geológicas. Uma maneira de levantar informações a respeito do comportamento tridimensional a partir da movimentação tectônica gerada pelo balanceamento das seções geológicas.

Inicialmente foi apresentado uma breve caracterização do que se trata a restauração de seções geológicas com enfoque maior naquilo que o Sistema Recon MS, base de desenvolvimento, oferece para realizar tal atividade. Dentre os recursos fornecidos pode-se citar a estrutura de dados topológicos HED, geração de malhas em seções, transformações geométricas e por fim, as linhas de mapeamento.

Este mapeamento realizado nas seções foi base para a criação das \emph{LMModels} e com isso se obteve o mapeamento das entidades geológicas ao longo da restauração. Ao exportar as \emph{LMModels} para um ambiente multisseções, foi possível associar este mapeamento às superfícies tridimensionais. A organização e uso desses dados de \emph{LMModels} serviram como parâmetros para, através de um método numérico, realizar uma deformação às superfícies a fim de submetê-las à mesma movimentação das seções. O responsável por essa deformação vem do processo de minimização de uma energia de alta ordem, tri-harmônica, que resulta em uma superfície de mínima variação.

Em relação ao volume, idealizou-se que ele fosse discretizado em uma nuvem de pontos contida no domínio do modelo. Esses pontos devem sofrer uma movimentação segundo o deslocamento dos pontos das seções de uma \emph{EtapaMS} a outra. A fim de melhorar essa movimentação, podem ser usados os pontos das superfícies que foram deformadas no mapeamento de superfícies em conjunto dos pontos das seções. A nuvem de pontos é criada a partir de uma grade volumétrica (cada célula da grade com 8 pontos). Os pontos das seções e superfícies, que possuem posição final definida, são os que movimentam os pontos da grade. 

\section{Mapeamento x Restauração}

A restauração de seções geológicas é uma atividade já consagrada na interpretação estrutural e fornece resultados satisfatórios de maneira eficiente. No entanto, ao considerar apenas seções transversais em modelos geológicos, que são tridimensionais, pode, em algum momento, acarretar na dissociação do comportamento real ocorrido. Em razão disso, pode-se partir a estudos que também levem em conta características 3D. Por outro lado, soluções assim podem ser custosas e ineficientes.

Este trabalho realizou uma busca e tratamento de informações provenientes da restauração de seções e as aplicou para deformar as superfícies e movimentar uma nuvem de pontos. Esse processo foi chamado de mapeamento pois trata-se de um levantamento e processamento de dados com o objetivo de caracterizar uma entidade geológica, assim como é feito no mapeamento geológico~\cite{Geoscan}. Neste caso, o objetivo foi realizar uma definição do comportamento das superfícies geológicas e do volume com os dados da restauração de seções.

Uma restauração de superfícies geológicas seria um processo análogo à restauração de seções, no entanto, com suas premissas geológicas e métodos próprios. É possível que nem sejam usadas seções transversais para esse intuito. E se incluir o volume para uma restauração do modelo 3D diretamente, mais questões acerca da metodologia precisam ser feitas. 

Uma restauração 3D deve levar em conta ainda mais aspectos, como parâmetros geomecânicos além dos geométricos. Alguns autores~\cite{Santi_3dgeological, Massot, DURANDRIARD2010441} já apresentaram propostas para esta finalidade. Propostas estas que incluem uma malha volumétrica, uso de elementos finitos, caracterização do material e definição de condições de contorno entre outros itens.

O que foi apresentado neste trabalho foi uma proposta inovadora cujo propósito se baseou na restauração 2D para se obter uma mapeamento de informações tridimensionais a custo relativamente baixo e computacionalmente eficiente.

\section{Observações}

A atividade chamada de fatiamento apresentada no item~\ref{item-section-creation} produz uma seção geológica a partir de um conjunto de superfícies tridimensionais. No entanto, a depender do estado do dado de entrada, as linhas de horizontes e falhas geradas podem não ser suficientes para que haja uma seção geológica apta para se realizar uma restauração. É comum que sejam feitas edições nas linhas como extensões e aparas a fim de fechar regiões e manter uma subdivisão planar contínua.

Estas edições acabam por gerar uma separação entre a superfície 3D e a seção bidimensional. Como as linhas de horizonte das seções são base das \emph{LMModels} usadas no mapeamento de superfícies, ao realizar a exportação dos dados para o deformador, podem ocorrer movimentações inesperadas e assim obter resultados não coerentes com a restauração de seções. Consequentemente, isto se estende à realização do mapeamento do volume.

\section{Propostas para trabalhos futuros}

Em primeiro lugar, sendo este trabalho uma maneira de adaptar informações 2D para um ambiente tridimensional, é muito oportuno que sejam desenvolvidos mecanismos e ferramentas que tenham como intuito manter uma ligação entre as duas visões do modelo. Isto irá trazer mais qualidade às informações presentes no modelo geológico.

A edição nas linhas pós-criação das seções vista no item anterior são sempre passíveis de acontecer. Como são nas seções que todo o trabalho de restauração ocorre, o modelo geológico deve ser guiado por estas alterações feitas pelos geólogos. Em outros termos, edições nas linhas das seções devem ser propagadas às superfícies para manter a coesão entre essas partes, uma vez que representam a mesma entidade geológica sob pontos de vista diferentes.

Em relação ao mapeamento do volume, como já citado, é dado a possibilidade de manter um acompanhamento de um ponto qualquer no domínio do modelo geológico ao longo de cada passo da restauração. Pode-se inclusive trabalhar na movimentação de uma seção geológica inteira a partir da restauração de seções vizinhas. Propõe-se ainda uma aplicação das transformações geológicas, que são usadas nas seções, no volume como uma maneira de movimentar a nuvem de pontos sem depender somente dos pontos das seções.
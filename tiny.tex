%% -*- coding: utf-8; -*-

\documentclass[
  master
  brazilian
]{ThesisPUC}


%%%
%%% Additional Packages
%%%

  \usepackage[brazilian]{babel}      %% in ThesisPUC.cls
  %% \usepackage[utf8]{inputenc}        %% .
  %% \usepackage[T1]{fontenc}           %% .
  %% \usepackage{lmodern}               %% .
  %% \usepackage[pdftex]{graphicx}	%% .

  \usepackage{tabularx}
  \usepackage{multirow}
  \usepackage{multicol}
  \usepackage{colortbl}
  \usepackage[%
    dvipsnames,
    svgnames,
    x11names,
    fixpdftex
  ]{xcolor}
  \usepackage{numprint}
  \usepackage{textcomp}
  \usepackage{booktabs}
  \usepackage{amsmath}
  \usepackage{enumitem}
  \usepackage{amssymb}
  \usepackage{textcomp}
% \usepackage{etoolbox}
  \usepackage{eqparbox}
  \usepackage{xparse}
  \usepackage{float}
  \usepackage[bottom]{footmisc}

%% numprint 
\npthousandsep{.}
\npdecimalsign{,}

%% ThesisPUC option
\tablesmode{fig} %% [nada, , tab ou figtab]
\abreviationsmode{none} %% [none ou use] %% Default is [use]


%%%
%%% Counters
%%%

%% uncomment and change for other depth values
%% \setcounter{tocdepth}{3}
%% \setcounter{lofdepth}{3}
%% \setcounter{lotdepth}{3}
%% \setcounter{secnumdepth}{3}


%%%
%%% New commands and other global definitions
%%%

\input{defs}

%%%
%%% Misc.
%%%

\usecolour{true}

%%%
%%% Titulos
%%%

\author{Vinicius da Silva Costa Almada}
\authorR{Almada, Vinicius da Silva Costa}

\advisor{Luiz Fernando Martha}{Prof.}
\advisorR{Martha, Luiz Fernando}
% If the advisor's department is different from author's department, uncomment the next line and type the correct name and acronym of advisor's institution.
\advisorInst{Departamento de Engenharia Civil}{PUC-Rio}

\coadvisor{André Luís Müller}{Dr.}
\coadvisorR{Müller, André Luís}
\coadvisorInst{Instituto Tecgraf}{PUC-Rio}

%% \title{Desenvolvimento de um sistema de microscopia digital para
%%  classificação automática de tipos de hematita em minério de ferro}

\title{Mapeamento de Superfície e Volume Baseado em Restauração de Seções Geológicas}

\titleuk{Mapping of Surface and Volume based on Geological Section Restoration}

%% \subtitulo{Aqui vai o subtitulo caso precise}

\day{06}
\month{Agosto}
\year{2021}

\city{Rio de Janeiro}
\CDD{XYZ.AB}
\program{Engenharia Civil}
\school{Centro Técnico Científico}
\university{Pontifícia Universidade Católica do Rio de Janeiro}
\uni{PUC-Rio}

%%%
%%% Jury
%%%

\jury{%
  \jurymember{Ana Paula de Meireles Reis Pelosi}{Dra.}
    {CENPES}{CENPES/Petrobras}
  \jurymember{Marcelo Gattass}{Prof.}
    {Departamento de Informática}{PUC-Rio}
  \jurymember{Márcio Rodrigues de Santi}{Dr.}
    {Instituto Tecgraf}{PUC-Rio}
  \jurymember{André Maués Brabo Pereira}{Prof.}
    {Universidade Federal Fluminense}{Universidade Federal Fluminense}
}

%%%
%%% Resume
%%%

\resume{%
  Bacharel em Engenharia Civil pelo Instituto Federal de Educação, Ciência e Tecnologia do Maranhão (IFMA), formou-se em 2018. Foi bolsista de programas de Iniciação Científica PIBITI – IFMA, com projeto de desenvolvimento de software para Mecânica dos Solos e um segundo na área de Dinâmica das Estruturas. Este último foi base para seu Trabalho de Conclusão de Curso, cujo objetivo foi desenvolver um software para análise dinâmica de estruturas sujeitas à carregamento sísmico. Desde o fim de 2019 atua no Instituto Tecgraf como bolsista no Grupo de Modelagem Digital em Geociências.}

%%%
%%% Acknowledgment (REMINDER TO SCHOLARSHIP STUDENTS. Do not forget to thank the agencies that supported your work.)
%%%

\acknowledgment{%
  \noindent Primeiro parágrafo de agradecimento ...
  \bigskip

  \noindent Segundo parágrafo de agradecimento ...
}

%%%
%%% Catalog prekeywords
%%%

\catalogprekeywords{%
  \catalogprekey{Engenharia Civil}%
  \catalogprekey{Engenharia}%
}

%%%
%%% Keywords
%%%


\keywords{%
  \key{mapeamento de superfície}
  \key{mapeamento de volume}
  \key{geologia estrutural}
  \key{superfície de mínima Variação}
}

\keywordsuk{%
  \key{surface mapping}
  \key{volume mapping}
  \key{structural geology}
  \key{minimum variation surface}
}

%%%
%%% Abstract
%%%

\abstract{%
  A restauração geológica busca reverter processos geológicos, partindo de uma região com sua geometria atual para sua configuração original, prévia à deformação. A restauração geológica de seções transversais é um dos principais recursos da indústria de óleo e gás para auxiliar na interpretação e validação. Em geral, processos geológicos acontecem de forma tridimensional. No entanto, a restauração 3D é complexa e cara e não faz parte do fluxo de trabalho tradicional que preza por soluções rápidas e eficientes como a restauração de seções 2D. Este trabalho apresenta uma metodologia e o desenvolvimento de ferramentas para mapear o movimento tridimensional baseado na restauração de seções geológicas. Esta metodologia divide o problema em duas etapas. O primeiro passo mapeia o movimento das seções para as superfícies do modelo com o uso de um deformador de superfícies. Na sequência, o movimento das seções junto do movimento das superfícies mapeiam o movimento do volume, aqui discretizado em uma nuvem de pontos. A solução numérica do primeiro passo realiza a movimentação das superfícies considerando pontos de controle, restrições impostas pelo movimento das seções transversais em conjunto com a minimização da função tri-harmônica a fim de produzir superfícies de variação mínima. O segundo passo faz a movimentação do volume baseado em pontos de controle dados pela movimentação das seções adicionadas ao movimento das superfícies obtidas no passo 1. A base de desenvolvimento para estes estudos é o Sistema Recon-MS, um software desenvolvido pela PETROBRAS em parceria com o Instituto Tecgraf/PUC-Rio, em que, dentre outros recursos, permite a restauração de modelos geológicos.
}

\abstractuk{%
  Geological restoration aims to reverse geological processes, starting from a region with its current geometry to its original configuration, prior to deformation. The geological restoration of cross-section is one of the oil and gas industry's key capabilities to aid interpretation and validation. In general, geological processes occur in a three-dimensional way. However, 3D restoration is complex and expensive and not part of the traditional workflow that emphasizes fast and efficient solutions such as restoring 2D sections. This work presents a methodology and the development of tools to map the three-dimensional movement based on the restoration of geological sections. This methodology divides the problem into two steps. The first step maps the movement of cross sections to model surfaces using a surface deformer. Next, the movement of the sections together with the movement of the surfaces map the movement of the volume, here discretized in a point cloud. The numerical solution of the first step performs the movement of surfaces considering control points, restrictions imposed by the movement of cross sections together with the minimization of the tri-harmonic function in order to produce surfaces with minimum variation. The second step is the movement of the volume based on control points given by the movement of the sections added to the movement of the surfaces obtained in step 1. The development basis for these studies is the Recon-MS System, a software developed by PETROBRAS in partnership with the Tecgraf/PUC-Rio Institute, which, among other resources, allows the restoration of geological models.}

%%%
%%% Dedication
%%%

%\dedication{%
 % Para minha mãe
%}

%%%
%%% Epigraph
%%%

%\epigraph{%
%  My beautifull epigraph
%}
%\epigraphauthor{Wassily Kandinsky}
%\epigraphbook{Regards sur le passé}

%%%
%%% Hyphenation
%%%

\hyphenation{PON-TI-FÍ-CIA}

%%%
%%%
%%% Quotes command
\newcommand{\quotes}[1]{``#1''}

%%%
%%% 
%%%

\begin{document}

  % -*- coding: utf-8; -*-

\chapter{Introdução}

This is the first chapter...



  % -*- coding: utf-8; -*-

\chapter{Processo de Restauração de Seções Geológicas}

\section{Sistema Recon MS}

\subsection{Introdução}

O ambiente no qual este trabalho é desenvolvido é o \textit{Sistema Recon MS}\cite{ReconTecgraf}, um software amplamente usado dentro da indústria de óleo e gás pela Petrobras e capaz de auxiliar na restauração de modelos geológicas. Conta com editor gráfico, estruturas de dados topológicos, algoritmos de transformações geológicas, gráficos de pós-processamentos entre outros recursos.

O Sistema Recon vem sendo desenvolvido a partir de um convênio entre o Instituto Tecgraf/PUC-Rio e a Petrobras desde 1991. Atualmente sua equipe responsável é formada pelo Grupo de Modelagem Digital em Geociências do Tecgraf. Uma imagem (Figura~\ref{fig-recon}) da tela inicial do programa é mostrada abaixo.

\begin{figure} [H]
  \begin{center}
    \includegraphics[width=\textwidth]{images/fig-recon}
    \caption{Captura de tela do Sistema Recon MS\cite{Recon}.}\label{fig-recon}
  \end{center}
\end{figure}

De acordo com Fossen\cite{Fossen}, restauração de seções geológicas pode ser entendida como uma manipulação da seção a fim de realizar a reconstituição dela ao seu estado anterior às deformações ocorridas ao longo do tempo, em outras palavras busca-se realizar uma retrodeformação na seção e utilizá-la na interpretação estrutural de uma região de interesse.

Neste capítulo são apresentadas as principais características do Sistema Recon MS para o objeto deste trabalho a fim de prover uma contextualização para o que é exibido nos demais capítulos. As próximas subseções tratam da descrição dos componentes principais e recursos básicos disponibilizados pelo sistema no processo de restauração de seções geológicas e também de visualização tridimensional do modelo. 

\subsection{Subdivisão Planar} % Falar do HED e da TopS

Uma seção geológica pode ser representada digitalmente por uma subdivisão planar uma vez que ela pode ser vista como um conjunto de polígonos que dividem o domínio da seção. Estes polígonos podem sofrer deformações e deslocamentos oriundos das transformações geológicos que a seção pode sofrer durante o balanceamento. Há ainda informações de adjacências entres essas porções que também precisam ser consideradas em um contexto computacional da seção geológica.

Na Figura~\ref{fig-subdivisao-planar} é possível perceber, por exemplo, que as camadas A, B e C possuem 3 blocos separados por falhas. Cada bloco é uma região fechada delimitada por um conjunto de segmentos. Deve-se observar ainda que essas regiões possuem atributos geológicos como idade, litologia, porosidade, etc.

\begin{figure} [h]
  \begin{center}
    \includegraphics[width=400pt]{images/fig-subdivisao-planar}
    \caption{Seção geológica como uma subdivisão planar.\cite{Ferraz}}\label{fig-subdivisao-planar}
  \end{center}
\end{figure}

Segundo Berg\cite{Berg}, uma subdivisão planar pode ser definida como uma subdivisão do plano através do uso de \textit{arestas}, \textit{vértices} e \textit{faces}. Essas são as entidades topológicas presentes em uma subdivisão planar, a face é como a região descrita anteriormente, delimitada por arestas (segmentos de curva); os vértices são os limites das arestas, sendo um para cada extremidade (podendo ser o mesmo vértice no início e no final da aresta).

A subdivisão planar precisa atender a alguns requisitos em relação às entidades topológicas: não deve haver vértices coincidentes; arestas só podem se cruzar em um vértice e faces também só se cruzam ou em um vértice, ou em uma aresta. Em outras palavras, não deve existir sobreposição de elementos topológicos. 

No entanto, há ainda um último componente topológico: o \textit{loop} ou \textit{laço} que é, de forma sucinta, um suconjunto conexo e ordenado de arestas. Com essa definição, a \textit{face} pode ser interpretada como uma união de laços, um deles sendo externo (delimitando a fronteira externa da face) e zero ou mais internos.

Em suma, a subdivisão planar tem os seguintes elementos topológicos:
\renewcommand{\labelitemi}{•}
\begin{itemize}
  \item \textbf{Vértice}: representa um ponto único dentro do plano.
  \item \textbf{Aresta}: segmento de curva com vértices como limites.
  \item \textbf{Laço} (loop): suconjunto conexo e ordenado de arestas.
  \item \textbf{Face}: região delimitada por um ou mais laços.
\end{itemize}

\subsection{Modelagem da Subdivisão Planar}

Para modelar a subdivisão planar dentro do Recon é utilizada a biblioteca computacional \textbf{HED} desenvolvida pelo Instituto Tecgraf/PUC-Rio e é a implementação de uma estrutura de dados topológicos baseada em arestas, a \textit{Half-Edge}\cite{HED}, uma das razões para esta escolha são as relações fixas de adjacência que uma aresta apresenta em relação às outras componentes topológicas. Uma aresta sempre é delimitada por dois vértices (distintos ou não) e é adjacente à duas faces.

O HED introduz uma nova entidade que explora bem essa característica denominada \textit{half-edge} ou \textit{semiaresta} que é uma referência ao \quotes{uso} da aresta por uma face. Dessa forma, no HED, cada aresta é formada por duas semiarestas. Cada semiaresta guarda uma referência para uma face e também para um vértice de origem. Isto dá uma orientação para a semiaresta que é usada para indicar o sentido positivo do loop das faces, por exemplo.

A estrutura HED tem um aspecto hierárquico de listas duplamente encadeada de elementos topológicos. No nível mais alto está a subdivisão planar, denominada como \textit{HedSolid}, então vêm \textit{HedFace}, \textit{HedLoop}, \textit{HedHalfEdge} e \textit{HedVtx} no nível mais baixo. A representação da aresta, \textit{HedEdge} encontra-se no mesmo nível da HedHalfEdge.

Uma propriedade importante em estruturas topológicas são as relações de adjacências entre suas componentes, a HED não provê de forma direta todas as relações, contudo é possível chegar às demais com uso de indireções. Por exemplo, partindo de uma aresta, como chegar às faces vizinhas? Basta ir às semiarestas da aresta, cada semiaresta possui referência para uma face.

Apresentado o HED e seus elementos, a associação com as entidades geológicas é intuitiva. Uma camada geológica é representada por uma face; as linhas de horizonte, falha ou sal têm como correspondente as arestas, por último, cada conjunto contínuo de faces é associado a um sólido\footnote{Os sólidos representam uma subdivisão planar e em alguns casos, a seção pode apresentar partes inteiramente descontínuas onde cada uma é um sólido diferente. Para casos onde é necessário sobreposição de partes, só é possível com a existência de mais de um sólido.}.

Destaca-se que a ideia de representar a seção geológica como uma subdivisão planar, ou uma estrutura HED, visa facilitar a criação e manipulação computacional da seção durante o processo de restauração. Todavia, a representação completa precisa levar em consideração também os atributos geológicos. Mais detalhes sobre a estrutura de dados HED podem ser encontrados em Mäntÿla~\cite{HED}, Arruda~\cite{Arruda} e Botsch \textit{et al.}~\cite{Botsch}.

\subsection{Atributos Geológicos}

Como já dito, os blocos que formam a seção geológica possuem propriedades próprias e precisam também estarem salvas na estrutura de dados topológica.

Cada entidade do HED possui um campo reservado para um tipo genérico de informações e neste espaço são organizados os atributos geológicos da seção. Estes atributos são representados em estruturas chamadas \textit{GeoSolid}, \textit{GeoFace}, \textit{GeoEdge} e \textit{GeoVtx}. Pela nomenclatura, é fácil observar a relação com o HED. As principais informações organizadas nessas estruturas são:

\renewcommand{\labelitemi}{•}
\begin{itemize}
  \item \textbf{GeoSolid}: o sólido por ser a estrutura de mais alto nível, é quem vai guardar referência à seção e ao cenário ao qual pertence dentro da restauração.
  \item \textbf{GeoFace}: é a estrutura que precisa armazenar dados do material geológico que a compõe (como idade, tipo, características físicas, etc.) e malha de triângulos que pode ser manipulada pelas transformações.
  \item \textbf{GeoEdge}: estrutura que guarda o tipo de linha (de horizonte, falha, topo de sal, etc.) e a subdivisão geométrica que forma a linha. 
  \item \textbf{GeoVtx}: é a única que armazena apenas o identificador universal.
\end{itemize}

Aliás, todas as estruturas de atributos geológicos possuem um campo para salvar este identificador que possui o formato \textit{UUID} --- \textit{universally unique identifier}~\cite{UUID} ou identificador único universal que é usado, por exemplo, na associação dos elementos geológicos com a malha triangular das faces, que será apresentada adiante.

\subsection{Seções Geológicas} % Falar da árvore de cenários

O principal recurso do Sistema Recon é seu conjunto de ferramentas para manipular uma seção geológica, desde a digitalização das informações que a definem geometricamente, da caracterização dos materiais e propriedades, da criação de dispositivos de controle e monitoramento da restauração até o kit de transformações que irão deformar a seção.

\subsubsection{Criação de uma seção geológica}

Para criar uma seção geológica no Sistema Recon pode-se recorrer ao editor gráfico para desenhar linhas e atribuir propriedades manualmente conforme seu tipo (se for horizonte, falha, limites da seção, etc.) ou em modelos que apresentem superfícies tridimensionais, como na Figura~\ref{fig-recon-1}, as seções podem ser criadas pela interseção de um plano vertical segundo uma direção dada pelo usuário, essa ação é chamada de \textit{fatiamento} do modelo.

\begin{figure} [H]
  \begin{center}
    \includegraphics[width=\textwidth]{images/fig-recon-1}
    \caption{Sistema Recon exibindo um modelo com superfícies tridimensionais e uma seção em destaque.}\label{fig-recon-1}
  \end{center}
\end{figure}

\subsubsection{Malhas da seção geológica}

A seção geológica é representada como uma subdivisão planar, como já citado, e é utilizada a biblioteca HED na implementação dessa subdivisão. Na Figura~\ref{fig-recon-2} pode-se observar uma seção geológica e alguns elementos, como as linhas (\textit{HedEdges}) onde a sua cor representa o atributo de tipo e as faces (\textit{HedFaces}) que são, em termos simples, regiões fechadas por linhas. Neste exemplo, todas as faces pertencem à mesma camada geológica.

\begin{figure} [h]
  \begin{center}
    \includegraphics[width=\textwidth]{images/fig-recon-2}
    \caption{Seção geológica com destaque para os elementos de linhas e faces.}\label{fig-recon-2}
  \end{center}
\end{figure}

As faces têm um atributo muito importante para o trabalho de restauração, que são as malhas de triângulos. Cada face possui uma malha independente das outras. No Sistema Recon, essa malha é armazenada numa estrutura de dados topológicos chamada \textit{TopS} que trata-se de uma biblioteca computacional voltada para representação de malhas de elementos finitos.\cite{Tops} A Figura~\ref{fig-recon-3} exibe a mesma seção, mas com adição das malhas das faces.

\begin{figure} [H]
  \begin{center}
    \includegraphics[width=350pt]{images/fig-recon-3}
    \caption{Malhas das faces de uma seção geológica no Sistema Recon.}\label{fig-recon-3}
  \end{center}
\end{figure}

A importância das malhas dentro da restauração de seções no Sistema Recon se dá por conta das transformações geológicas que possuem como requisito de entrada uma malha. Um pouco mais de detalhes sobre as transformações será apresentado adiante.

A estrutura \textit{TopS} permite armazenar certos atributos em seus elementos topológicos. Em especial nos vértices da malha, no Sistema Recon, é armazenado o \textit{UUID} do atributo geológico da entidade topológica do HED sobre a qual aquele vértice está, em outras palavras, se o vértice da malha está no interior da face, ele guarda o \textit{UUID} da \textit{GeoFace} dessa face, o mesmo para caso esteja sobre uma aresta (\textit{GeoEdge}) ou vértice (\textit{GeoVertex}). A Figura~\ref{fig-recon-4} mostra um exemplo da forma como esses dados são obtidos.

\begin{figure} [H]
  \begin{center}
    \includegraphics[width=\textwidth]{images/fig-recon-4}
    \caption{Trecho de uma malha de face com os tipos de atributo geológicos que estão sob os vértices da malha.}\label{fig-recon-4}
  \end{center}
\end{figure}

Essa relação permite identificar, a partir de um vértice da malha, sobre qual entidade geológica está este vértice, este recurso será usado no próximo capítulo.

\subsubsection{Transformações}

As transformações geológicas são ferramentas que buscam reverter (ou simular) as movimentações e deformações ocorridas ao longo do tempo.\cite{Santi} As transformações são aplicadas diretamente às malhas, no entanto, para que isso aconteça, é necessário antes a definição de \textit{Módulos} na seção. 

Módulos são agrupamentos de faces ou blocos da seção, têm o intuito de reunir aquelas partes que devem ter recebido as mesmas deformações, podendo ser, inclusive partes de camadas diferentes.

Com o módulo definido, consegue-se aplicar uma transformação. Esta irá ser empregada sobre a malha de cada uma das faces que compõem aquele módulo, deformando esta malha e, por conseguinte, alterando a geometria da seção.

A Figura~\ref{fig-recon-5} mostra a aba \quotes{Transformações} do Sistema Recon onde é possível ver os grupos de transformações e também as opções disponíveis. Mais detalhes sobre cada uma delas podem ser consultados no manual do usuário do programa\cite{Recon}.

\begin{figure} [H]
  \begin{center}
    \includegraphics[width=\textwidth]{images/fig-recon-5}
    \caption{Aba \quotes{Transformações} do Sistema Recon.}\label{fig-recon-5}
  \end{center}
\end{figure}

\subsubsection{Árvore de cenários}

A restauração de seções é um processo linear no sentido de que cada novo passo depende de como estava o anterior. Qualquer mudança num passo desses acarreta em um resultado diferente ao final. Além do mais, balanceamento de seções não é uma atividade de resposta única, o objetivo é obter uma interpretação geológica restaurável, ou seja, seus componentes estruturais devem poder ser restauradas~\cite{Fossen}. 

Diante disto, o Sistema Recon disponibiliza em sua interface de manipulação das seções um componente capaz de registrar o histórico de etapas no processo de restauração, mais que isso, ao usuário é dada a possibilidade de voltar em algum ponto e criar uma nova linha de estudo dentro desse processo, ou ainda apagar uma sequência de etapas que ele julga estar incorreta.

Isso tudo é possível graças à árvore de cenários. Um cenário é a representação de um estado de restauração de uma seção. Por exemplo, se de um passo a outro da restauração ocorre uma transformação, o estado anterior pode ser registrado em um cenário e o posterior em um outro. De cada cenário pode-se criar diversos outros como se fossem diferentes linhas do tempo, ou diferentes interpretações daquele passo de restauração.

Árvores são um tipo especial de estrutura de dados não-linear e neste casos de uso é definida como tendo uma raiz ou nó inicial que aponta para um ou mais nós. Estes, igualmente, podem apontar para diferentes nós numa escala hierárquica. A Figura~\ref{fig-recon-6} apresenta um exemplo de árvore de cenários tirada do Sistema Recon. Nesta imagem, cada quadrinho representa a seção num dado estado e como identificação, cada cenário também possui um número.

\begin{figure} [H]
  \begin{center}
    \includegraphics[width=160pt]{images/fig-recon-6}
    \caption{Exemplo de árvore de cenários de uma seção do Sistema Recon.}\label{fig-recon-6}
  \end{center}
\end{figure}

O primeiro cenário tem a seção em sua versão inicial e a cada nova manipulação da mesma, pode-se criar um novo cenário e assim ter este histórico. Essa maneira de organizar uma restauração é útil não só no contexto de uma seção isolada, mas principalmente quando se trabalha em modelos de multisseções que irão sofrer os mesmos processos de restauração, mas de maneiras diferentes. Com um registro do quê e quando ocorreu uma dada transformação em diferentes seções é possível ter uma visão mais geral do modelo em uma sequência cronológica.

\subsection{Ambiente Multisseções}

Apesar dos principais recursos do Sistema Recon atuarem diretamente com seção geológica, isso não significa dizer que só seja possível manipular modelos com uma única seção geológica. Uma das grandes mudanças ocorridas no programa foi a criação de ferramentas para se tratar de modelos com múltiplas seções, ou modelos multisseção~\cite{Felipe},~\cite{Garcia}.

O ambiente multisseção (MS) do Sistema Recon trata-se de um visualizador 3D onde podem ser vistas as superfícies geológicas e também as seções em um contexto global do modelo.

Como sistema de referências, o ambiente MS usa coordenadas UTM (Universal Transversa de Mercador)~\cite{IBGE} para localizar seus objetos. Neste sistema, cada ponto é representado por um par $(N, E)$ onde $N$ é a coordenada norte-sul em metros e $E$, a coordenada leste-oeste.

A Figura~\ref{fig-recon-7} exibe o Sistema Recon no ambiente MS, onde é possível notar o (1) visualizador tridimensional com superfícies e seções geológicas, (2) a lista de \textit{EtapasMS} que será apresentada a seguir juntamente da (3) lista de cenários da etapa.

\begin{figure} [H]
  \begin{center}
    \includegraphics[width=\textwidth]{images/fig-recon-7}
    \caption{Ambiente Multisseção do Sistema Recon.}\label{fig-recon-7}
  \end{center}
\end{figure}

\subsection{Etapas de restauração}

Como brevemente apresentado, o ambiente MS permite ter uma olhar mais global do modelo e de todos os componentes que o formam. Neste contexto é então preciso organizar as seções de forma que haja o máximo de coerência do ponto de vista geral durante a restauração do modelo. Podem existir seções que compartilham uma mesma falha ou um mesmo evento tectônico, por exemplo.

Como já visto, cada seção conta com um registro de cada passo dado no andamento da restauração e é um recurso presente apenas localmente e independente. No entanto, seções relativamente próximas, ou que foram restauradas de maneira semelhante precisam sincronizar esse histórico para que haja uma ordem melhor do modelo sob um aspecto global.

Com essa finalidade, foi criado e implementado o conceito de \textit{EtapaMS}. Uma \textit{EtapaMS} trata-se de um conjunto de cenários de seções diferentes mas que, de certa forma, representam o mesmo marco geológico, como a restauração de uma falha ou descompactação. Cada \textit{EtapaMS} pode ter apenas 1 cenário por seção dentro de sua estrutura, isso permite ter um histórico do modelo multisseção análogo à árvore de cenário da seção individualmente.

No Sistema Recon, as \textit{EtapasMS} são dispostas em lista no ambiente multisseção. Ao selecionar um item dessa lista, logo abaixo é exibido o conjunto de cenários (por seção) que compõem aquela \textit{EtapaMS}, como bem mostra a Figura~\ref{fig-recon-7}.

Uma forma de uso das \textit{EtapasMS} para a restauração de modelos geológicos é organizar os estados de seções diferentes que respondam ao mesmo evento ou marco geológico. Caso haja uma falha X que atravessa 3 seções e em todas elas essa falha é restaurada, pega-se o cenário de cada seção onde isso ocorre e cria-se uma \textit{EtapaMS} correspondente a este marco. Esta organização da restauração é parte importante para o mapeamento de superfícies e volume.



  % -*- coding: utf-8; -*-

\chapter{Linhas de Mapeamento}

O mapeamento descrito neste trabalho é baseado exclusivamente na restauração de seções geológicas. Para tanto, há a necessidade de uma camada de informações proveniente das seções que contenha os dados a serem usados no mapeamento tridimensional. Essas informações podem ser extraídas com auxílio de um objeto geométrico auxiliar presente nas seções geológicas: \textit{linhas de mapeamento}.

Neste capítulo é apresentado o conceito de linha de mapeamento presente no Sistema Recon, suas características, alguns casos de uso e também suas derivações.

\section{Conceito}

Linhas de mapeamento são linhas formadas por \textit{pontos de mapeamento}. Estes pontos são objetos mapeados nas malhas da seção e guardam a informação referente à sua localização dentro desta malha.

Um ponto de mapeamento, em razão dos tipos de entidades topológicas presentes na malha, pode ser do tipo nó, aresta ou elemento:

\renewcommand{\labelitemi}{•}
\begin{itemize}
  \item Nó: o ponto está sobre um nó da malha. É guardado o identificador desse nó.
  \item Aresta: caso onde o ponto localiza-se sobre uma aresta de elemento. Além do identificador da aresta, é armazenada a coordenada paramétrica do ponto na aresta.
  \item Elemento: o ponto encontra-se no interior de um elemento. Guarda-se o identificador do elemento e as coordenadas baricêntricas do ponto no elemento triangular.
\end{itemize}

A criação dessas linhas se dá pela definição de uma linha-guia que pode cruzar diferentes regiões da seção. Para cada região interceptada, é criada uma parte de linha de mapeamento, essa parte armazena o identificador da malha da região. A interseção dos pontos da linha-guia com a malha produz os pontos de mapeamento.

A linha de mapeamento pode ser criada em qualquer cenário durante a restauração da seção e sua geometria pode ser calculada com base na malha em função dos pontos de mapeamento que a formam. Após a criação, uma versão da linha de mapeamento é gerada para cada cenário anterior e subsequente ao que foi usado na definição da linha-guia. Com isso, pode ser realizado o mapeamento dessa linha ao longo das etapas de restauração da seção.

O requisito para que seja calculada a geometria da linha em diferentes cenários é que a malha mantenha a mesma topologia. No entanto, mesmo em casos de edição, é possível realizar uma interpolação dos atributos presentes na malha para sua nova versão. Incluem-se nisso as partes de linha de mapeamento que irão também receber uma nova versão equivalente dada a alteração na topologia da malha.

Dentro do Sistema Recon MS, a linha de mapeamento é um recurso importante na interpretação dos resultados gerados na restauração do modelo. Com ela é possível ter uma linha que acompanha a movimentação da malha de um cenário a outro.

As linhas de mapeamento (Figura~\ref{fig-linemap}) permitem realizar um mapeamento geométrico ao longo de uma restauração tomando como base uma linha-guia poligonal definida pelo usuário.

\begin{figure} [h]
  \begin{center}
    \includegraphics[width=\textwidth]{images/fig-linhas-de-mapeamento-ed}
    \caption{Linhas de mapeamento em uma seção.}\label{fig-linemap}
  \end{center}
\end{figure}

A Figura~\ref{fig-linemap-history} apresenta o resultado após uma transformação do tipo \textit{move sobre falha} (MSF)~\cite{Recon} onde é possível observar, além da deformação da camada, a linha de mapeamento sofrendo a mesma movimentação. Este tipo de uso pode ser interpretado como se houvesse ali um falso horizonte para avaliar o quantidade de movimento na restauração do rejeito.

\begin{figure} [h]
  \begin{center}
    \includegraphics[width=\textwidth]{images/fig-linemap-history}
    \caption{Linhas de mapeamento em diferentes etapas}\label{fig-linemap-history}
  \end{center}
\end{figure}

\section{Criação das Linhas de Mapeamento}

O processo de criação da linha de mapeamento é feito para cada parte individualmente, de forma que, ao visualizar as partes tem-se a linha de mapeamento completa. Na Figura~\ref{fig-linemap-malhas} é possível ver uma linha de mapeamento cortando algumas regiões diferentes.

\begin{figure} [h]
  \begin{center}
    \includegraphics[width=250pt]{images/fig-linhas-de-mapeamento-malhas}
    \caption{Linhas de mapeamento cortando múltiplas faces.}\label{fig-linemap-malhas}
  \end{center}
\end{figure}

Na Figura~\ref{fig-linemap-parts} estão evidenciadas as partes que formam a linha de mapeamento. Como já dito, cada parte está associada à malha de um região diferente.

\begin{figure} [h]
  \begin{center}
    \includegraphics[width=250pt]{images/fig-lm-parts}
    \caption{Partes de uma linha de mapeamento}\label{fig-linemap-parts}
  \end{center}
\end{figure}

A Figura~\ref{fig-lm-topo} mostra a identificação dos pontos em uma parte de linha de mapeamento e a Tabela~\ref{tab-lm-topo} exibe quais informações topológicas são salvas de cada ponto.

\begin{figure} [hbt!]
  \begin{center}
    \includegraphics[width=260pt]{images/fig-lm-topo}
    \caption{Informações topológicas da malha mapeadas para a linha de mapeamento.}\label{fig-lm-topo}
  \end{center}
\end{figure}

% -*- coding: utf-8; -*-

\begin{table} [hbt!]
 \begin{center}
	 \caption{Informações topológicas salvas na linha de mapeamento.\label{tab-lm-topo}}
	~\\[-2mm]
	 \begin{tabularx}
		 {\textwidth}
		 {cp{2.0cm} lp{3.0cm} lp{10.0cm}}

		 \textbf{Ponto}
		 & \textbf{Tipo}
		 & \textbf{Informação armazenada} \\ \toprule

		 %~\\[-1mm]
		 A
		 & Elemento
		 & id=30, coordenadas baricêntricas=(0,33; 0,33; 0,33) \\ \midrule

		 %~\\[-1mm]
		 B
		 & Nó   
		 & id=431 \\ \midrule

		 %~\\[-1mm]
		 C
		 & Aresta
		 & id=130, coordenada paramétrica=0,45 \\ \midrule

		 %~\\[-1mm]
		 D
		 & Aresta
		 & id=145, coordenada paramétrica=0,55 \\ \midrule

	 \end{tabularx}
 \end{center}
\end{table}


\section{Derivações das Linhas de Mapeamento}

As linhas de mapeamento têm também casos de usos mais especializados dentro do Sistema Recon, como na criação e representação de poços. Poços são criados semelhantemente às linhas ou por importação de modelos com poços em 3D. Possuem característica de serem linhas quase verticalizadas e possuem uma finalidade mais limitada. Nos casos de poços 3D, a linha correspondente ao poço é apenas uma projeção do objeto tridimensional no plano da seção.

Há o uso nas chamadas linhas de interseção (\textit{CrossLine}) que servem para identificar e mapear as linhas de cruzamento entre seções no espaço tridimensional do multisseções, com isso é possível ter uma noção do que ocorre com seções transversais mesmo estando no domínio bidimensional da restauração.

Por último, foi criada a \textit{linha de mapeamento do modelo} ou \textit{LMModel}, cujo objetivo é servir para o mapeamento de linhas geológicas das seções para superfícies de horizontes geológicos e falhas. As \emph{LMModels} representam os elementos geológicos ao longo da restauração do modelo. Assim, é possível ter um acompanhamento do que ocorre com as entidades geológicas na seção, além de poder verificar como se deu a movimentação de cada ponto de horizonte, falha ou topo de sal ao longo da restauração.

As \textit{LMModels} são linhas de mapeamento baseadas no pontos do contorno da malha das regiões, ou seja, as partes que a formam possuem apenas pontos de mapeamento do tipo nó.

Pelo objetivo proposto, as \textit{LMModels} são linhas de mapeamento que tomam a geometria das entidades geológicas como entrada. Então, não há necessidade de criar uma linha-guia como é feita na linha de mapeamento original; a própria linha de horizonte, falha, ou topo de sal é usada como linha-guia.

Conforme o tipo do elemento geológico base, há um tipo de \textit{LMModel} e informações adicionais armazenadas:

\renewcommand{\labelitemi}{•}
\begin{itemize}
  \item Horizonte: a informação de idade deste horizonte.
  \item Falha: o identificador da falha é o dado armazenado.
  \item Topo de sal: apenas uma referência direta à linha original.
\end{itemize}

Todas essas informações  geológicas atreladas ao mapeamento topológico das \textit{LMModels}, quando em conjunto com as diversas seções geológicas de um modelo multisseções, são o que fazem dela o principal dado para a realização de um mapeamento de informações de evolução do modelo tridimensional ao longo do tempo, já que trazem todo o histórico de movimentação das camadas de um modelo geológico.

A utilização das \emph{LMModels} neste trabalho é feito com o auxílio de estruturas de dados que organizam as informações em subconjuntos divididos por etapa de restauração e idade (caso de linhas de horizonte), o que é visto na sequência. Com isso, é obtido o conjunto de informações que representam a restauração das seções no Sistema Recon.

\begin{figure} [h]
  \begin{center}
    \includegraphics[width=350pt]{images/fig-lmmodel-example}
    \caption{\textit{LMModels} em uma seção geológica no Sistema Recon}\label{fig-lmmodel-example}
  \end{center}
\end{figure}

A Figura~\ref{fig-lmmodel-example} mostra \textit{LMModels} de dois horizontes diferentes. A representação delas dentro do Sistema Recon é feita com uma linha de maior espessura que as linhas de horizonte. No entanto, os pontos são os mesmos do contorno da malha. Observa-se na Figura~\ref{fig-lmmodel-mesh-diff} a diferença entre a linha de horizonte e a \textit{LMModel}, evidenciando o contorno da malha entre as duas regiões.

\begin{figure} [h!]
  \begin{center}
    \includegraphics[width=275pt]{images/fig-lmmodel-mesh-diff}
    \caption{Diferença entre \textit{LMModel} e a linha de horizonte e uma seção geológica no Sistema Recon}\label{fig-lmmodel-mesh-diff}
  \end{center}
\end{figure}

A ilustração na Figura~\ref{fig-lmmodel-ms} apresenta as \textit{LMModels} de dois horizontes em um modelo multisseções. É esse conjunto de informações no ambiente tridimensional que será usado como parâmetro para o mapeamento de superfícies no Sistema Recon.

\begin{figure} [h!]
  \begin{center}
    \includegraphics[width=300pt]{images/fig-lmmodel-ms}
    \caption{\textit{LMModels} no ambiente multisseções do Sistema Recon}\label{fig-lmmodel-ms}
  \end{center}
\end{figure}



  \input{chapter-map-surf}
  % -*- coding: utf-8; -*-

\section{Mapeamento do Volume}

O mapeamento do volume geológico é feito com o uso uma nuvem de pontos contida no domínio tridimensional do modelo. Cada ponto precisa ter informação sobre a qual camada geológica ele pertence. O mapeamento é feito com a movimentação desses pontos a cada \textit{EtapaMS} de restauração. As fontes da movimentação são os nós das malhas das seções em conjunto dos pontos do mapeamento de superfície.

Nesta seção serão apresentados a metodologia por trás do mapeamento do volume e o sua aplicação dentro do Sistema Recon, bem como detalhes sobre os requisitos para utilização do método.

\subsection{Metodologia}

Nas últimas décadas vários métodos numéricos vêm sendo desenvolvidos para simulação de problemas físicos em modelos tridimensionais, dentre os quais destacam-se alguns, divididos em dois grupos e listados a seguir:

\renewcommand{\labelitemi}{•}
\begin{itemize}
  \item Métodos baseados em malhas tridimensionais: métodos de elementos finitos (FEM)\cite{MEF}; métodos de diferenças finitas (FDM)\cite{MDF}; métodos de volumes finitos (FVM)\cite{MVF}; métodos lagrangeanos euleriano arbitrário (ALE)\cite{ALE}.
  \item Métodos não baseados em malhas tridimensionais: método de partículas em células (PIC)\cite{PIC}; smooth particle hydrodynamics (SPH)\cite{SPH}; método de ponto material (MPM)\cite{MPM}; método de elementos discretos (DEM)\cite{DEM}; método de ponto material com interpolação generalizada (GIMP)\cite{GIMP,MullerGIMP}.
\end{itemize}

De maneira genérica, o problema físico-matemático para o mapeamento do volume baseado em restauração de seções e mapeamento de superfícies pode ser descrito por: dada a movimentação de um conjunto de seções e a movimentação de um conjunto de superfícies, movimentar um volume de forma compatível, respeitando restrições de movimentação nas falhas. Todos os métodos acima citados, além de outros não referenciados nesta lista, apresentam vantagens e desvantagens e poderiam ser utilizados para solucionar este problema. Todavia, a escolha de um método numérico deve contemplar as características do problema que por vezes demanda simplificações e/ou aprimoramentos além da avaliação da viabilidade de uso, por exemplo: tempo para simulação, geração de dados de entrada, estabilidade numérica, entre outros. De forma geral, modelos geológicos tridimensionais apresentam características de grande complexidade geométrica. Essa complexidade, por sua vez, acarreta grandes desafios para geração de malhas, especialmente quando devem ser consideradas restrições de horizontes e falhas. Além disso, de forma geral em Geologia, as etapas de modelagem e simulação tridimensionais apresentam soluções caras computacionalmente. Tendo em vista estes aspectos, Muller~\cite{Muller} desenvolveu uma metodologia de solução do problema matemático exposto anteriormente, baseada principalmente nos métodos GIMP e ALE, apresentada a seguir.

O problema físico-matemático é caracterizado pela movimentação do volume, que por sua vez é guiado pela movimentação de seções transversais e/ou superfícies. Dessa forma, as equações de conservação de massa e conservação de momento são suficientes para a solução do problema, Eqs.~\ref{eq-vol-1} e~\ref{eq-vol-2} respectivamente. Na conservação de momento, devido às características do problema, o termo relativo às forças de corpo (gravitacionais) pode ser desconsiderado.

\begin{align}
  &\frac{d\rho}{dt} + \rho \mathbf{\nabla}\cdot\boldsymbol{v} = 0\label{eq-vol-1}
\end{align}

\begin{align}
  &\rho\frac{d\boldsymbol{v}}{dt} - \mathbf{\nabla} \cdot \boldsymbol{\sigma} = 0\label{eq-vol-2}
\end{align}

Nas equações acima $\rho$ representa densidade, $\boldsymbol{v}$ velocidade e $\boldsymbol{\sigma}$ o tensor de tensões de Cauchy. Para evitar a necessidade de geração de malhas tridimensionais, a solução das Eqs.~\ref{eq-vol-1} e~\ref{eq-vol-2} será desenvolvida para um meio representado por pontos discretos arbitrários, em suas coordenadas $\boldsymbol{x}_p$, que por sua vez representam o volume $\Omega$ do modelo. A conectividade entre esses pontos será dada por uma grade volumétrica, em suas coordenadas $\boldsymbol{x}_i$.

A massa de cada ponto arbitrário pode ser descrita por:

\begin{align}
  &m_p =\int_{\Omega}\rho(\boldsymbol{x}_p)\chi(\boldsymbol{x}_p)\,d\Omega.
\end{align}

Sendo $\rho(\boldsymbol{x}_p)$ o campo de densidades e $\chi(\boldsymbol{x}_p)$ funções características. Para caracterizar a ligação entre os pontos e a grade são utilizadas funções peso $\boldsymbol{N}_{ip}$. Uma função $\boldsymbol{N}_{ip}$ representa o peso para o nó $i$ da grade avaliado na posição do ponto $p$, $\boldsymbol{N}_{ip} = \boldsymbol{N}(\boldsymbol{x}_p-\boldsymbol{x}_i)$ definido por:

\begin{align}
  &\boldsymbol{N}_{ip} = \frac{\displaystyle\int_{\Omega}\chi(\boldsymbol{x})\boldsymbol{\Phi}_i(\boldsymbol{x})\,d\Omega}{\displaystyle\int_{\Omega}\chi(\boldsymbol{x})\,d\Omega}.
\end{align}

Onde $\boldsymbol{\Phi}_i(\boldsymbol{x})$ é uma função de forma para o nó $i$. O gradiente da função peso pode ser definido por:

\begin{align}
  &\nabla\boldsymbol{N}_{ip} = \frac{\displaystyle\int_{\Omega}\chi(\boldsymbol{x})\nabla\boldsymbol{\Phi}_i(\boldsymbol{x})\,d\Omega}{\displaystyle\int_{\Omega}\chi(\boldsymbol{x})\,d\Omega}.
\end{align}

As funções características e de forma utilizadas na presente formulação são:

\begin{align}
  &\chi(\boldsymbol{x_p}) = 
    \begin{cases}
      1 &\text{se } |\boldsymbol{x_p}| < \cfrac{l}{2}\\
      0 &\text{em outro caso}
    \end{cases}\label{eq-vol-carac-1}
\end{align}


\begin{align}
  &\Phi(\boldsymbol{x_p}) = 
    \begin{cases}
      1 - \cfrac{4\boldsymbol{x_p}^2+l}{4hl}& \text{ se } |\boldsymbol{x_p}| < \cfrac{l}{2}\\
      \\
      1 - \cfrac{|\boldsymbol{x_p}|}{h}& \text{ se } \cfrac{l}{2} \leqslant |\boldsymbol{x_p}| < h - \cfrac{l}{2}\\
      \\
      \cfrac{\left(h+\cfrac{l}{2}-|\boldsymbol{x_p}|\right)^2}{2hl}& \text{ se } h-\cfrac{l}{2} \leqslant |\boldsymbol{x_p}| < h + \cfrac{l}{2}\\
      \\
      0 & \text{ em outro caso}
    \end{cases}\label{eq-vol-carac-2}
\end{align}

Nas Eqs.~\ref{eq-vol-carac-1} e~\ref{eq-vol-carac-2} $h$ representa o tamanho da célula da grade e $l$ um tamanho característico definido inicialmente em função do número de pontos contidos na célula.

Como já mencionado, o volume é representado por pontos arbitrários. De forma semelhante, as seções e superfícies também serão representadas por pontos, entretanto para esses, são conhecidas suas posições iniciais e finais em cada etapa. A movimentação desses pontos, levada para os nós da grade será a responsável pela movimentação dos pontos do volume. A movimentação dos pontos de seções e superfícies pode ser convertida numa velocidade adimensional em relação ao tempo. Após isso, para um determinado tempo $t$ avalia-se em cada nó $i$ da grade a massa, momento e velocidade, respectivamente como segue:

\begin{align}
  &m_i^t = \textstyle\sum_p\boldsymbol{N}_{ip}m_p^t
\end{align}

\begin{align}
  &\boldsymbol{p}_i^t = \textstyle\sum_p\boldsymbol{N}_{ip}m_p^t\boldsymbol{v}_p^t
\end{align}

\begin{align}
  &\boldsymbol{v}_i^t = \boldsymbol{p}_i^t / m_i^t
\end{align}

$\sum_p$ representa a soma sobre todos os pontos que possuem contribuição para uma determinada célula da grade e $\sum_i$ representa a soma sobre todos os nós da grade que contribuem para um determinado ponto.

O passo seguinte consiste numa etapa de advecção Euleriana simples que definirá a nova posição dos pontos do volume baseada nas informações que previamente foram levadas aos nós da grade. Para isto é avaliado o gradiente da velocidade de cada ponto, o tensor gradiente de deformação de cada ponto $\boldsymbol{F}_p$, que possibilitará o cálculo de deformações no volume e a posição atualizada de cada ponto.

\begin{align}
  &\boldsymbol{v}_p^{t+\Delta t} = \textstyle\sum_i\boldsymbol{N}_{ip}\boldsymbol{v}_i^{t+\Delta t}
\end{align}

\begin{align}
  &\boldsymbol{F}_p^{t+\Delta t} = (\boldsymbol{I} + \nabla\boldsymbol{v}_p^{t+\Delta t}\Delta t) \boldsymbol{F}_p^t
\end{align}

\begin{align}
  &\boldsymbol{x}_p^{t+\Delta t} = (\boldsymbol{I} + \nabla\boldsymbol{v}_p^{t+\Delta t}\Delta t) \boldsymbol{F}_p^t
\end{align}

Os incrementos de tempo considerados para integração temporal deste problema, virtualmente transiente, consideram as condições de Courant-Friedrichs-Lewy (CFL)~\cite{CFL}, garantindo assim a estabilidade numérica da solução.

Como mencionado anteriormente, as seções e superfícies do modelo são responsáveis pela informação da movimentação que guiará a movimentação do volume. Essa movimentação, no espaço das seções e superfícies, honrou restrições de movimento dadas pelas falhas geológicas. Evidentemente, se deseja que a movimentação do volume também respeite essas restrições, ou seja, as superfícies das falhas devem ser consideradas como restrições para a movimentação do volume. Para isso, as superfícies de falha também são discretizadas por pontos. Aos pontos do volume que estarão em contato com as falhas são impostas condições especiais de movimentação. Se considerarmos em um nó $i$ da grade a velocidade de um ponto do volume $\boldsymbol{v}_i^v$ e a velocidade de um ponto da falha $\boldsymbol{v}_i^f$ teremos contato se $\left(\boldsymbol{v}_i^v-\boldsymbol{v}_i^f\right)\boldsymbol{n}_i>0$. Sendo $\boldsymbol{n}_i$ a normal da superfície da falha calculada em $i$. Se a condição de contato é satisfeita e a condição de momento garantida, as velocidade podem ser corrigidas por:

\begin{align}
  &\bar{\boldsymbol{v}}_i^v = \boldsymbol{v}^v_i - m_i^f(\boldsymbol{v}^v_i - \boldsymbol{v}^f_i)\boldsymbol{n}_i \cfrac{\boldsymbol{n}_i}{(m_i^v+m_i^f)}
\end{align}

\begin{align}
  &\bar{\boldsymbol{v}}_i^f = \boldsymbol{v}^f_i - m_i^v(\boldsymbol{v}^v_i - \boldsymbol{v}^f_i)\boldsymbol{n}_i \cfrac{\boldsymbol{n}_i}{(m_i^v+m_i^f)}
\end{align}

\subsection{Preparação dos dados}

\subsection{Exemplos e resultados}



  % -*- coding: utf-8; -*-

\chapter{Conclusão}

\section{Resumo}

Este trabalho propôs um mapeamento do comportamento de superfícies e volume baseado na restauração de seções geológicas. Uma maneira de levantar informações a respeito do comportamento tridimensional a partir da movimentação tectônica gerada pelo balanceamento das seções geológicas.

Inicialmente foi apresentado uma breve caracterização do que se trata a restauração de seções geológicas com enfoque maior naquilo que o Sistema Recon MS, base de desenvolvimento, oferece para realizar tal atividade. Dentre os recursos fornecidos pode-se citar a estrutura de dados topológicos HED, geração de malhas em seções, transformações geométricas e por fim, as linhas de mapeamento.

Este mapeamento realizado nas seções foi base para a criação das \emph{LMModels} e com isso se obteve o mapeamento das entidades geológicas ao longo da restauração. Ao exportar as \emph{LMModels} para um ambiente multisseções, foi possível associar este mapeamento às superfícies tridimensionais. A organização e uso desses dados de \emph{LMModels} serviram como parâmetros para, através de um método numérico, realizar uma deformação às superfícies a fim de submetê-las à mesma movimentação das seções. O responsável por essa deformação vem do processo de minimização de uma energia de alta ordem, tri-harmônica, que resulta em uma superfície de mínima variação.

Em relação ao volume, idealizou-se que ele fosse discretizado em uma nuvem de pontos contida no domínio do modelo. Esses pontos devem sofrer uma movimentação segundo o deslocamento dos pontos das seções de uma \emph{EtapaMS} a outra. A fim de melhorar essa movimentação, podem ser usados os pontos das superfícies que foram deformadas no mapeamento de superfícies em conjunto dos pontos das seções. A nuvem de pontos é criada a partir de uma grade volumétrica (cada célula da grade com 8 pontos). Os pontos das seções e superfícies, que possuem posição final definida, são os que movimentam os pontos da grade. 

\section{Mapeamento x Restauração}

A restauração de seções geológicas é uma atividade já consagrada na interpretação estrutural e fornece resultados satisfatórios de maneira eficiente. No entanto, ao considerar apenas seções transversais em modelos geológicos, que são tridimensionais, pode, em algum momento, acarretar na dissociação do comportamento real ocorrido. Em razão disso, pode-se partir a estudos que também levem em conta características 3D. Por outro lado, soluções assim podem ser custosas e ineficientes.

Este trabalho realizou uma busca e tratamento de informações provenientes da restauração de seções e as aplicou para deformar as superfícies e movimentar uma nuvem de pontos. Esse processo foi chamado de mapeamento pois trata-se de um levantamento e processamento de dados com o objetivo de caracterizar uma entidade geológica, assim como é feito no mapeamento geológico~\cite{Geoscan}. Neste caso, o objetivo foi realizar uma definição do comportamento das superfícies geológicas e do volume com os dados da restauração de seções.

Uma restauração de superfícies geológicas seria um processo análogo à restauração de seções, no entanto, com suas premissas geológicas e métodos próprios. É possível que nem sejam usadas seções transversais para esse intuito. E se incluir o volume para uma restauração do modelo 3D diretamente, mais questões acerca da metodologia precisam ser feitas. 

Uma restauração 3D deve levar em conta ainda mais aspectos, como parâmetros geomecânicos além dos geométricos. Alguns autores~\cite{Santi_3dgeological, Massot, DURANDRIARD2010441} já apresentaram propostas para esta finalidade. Propostas estas que incluem uma malha volumétrica, uso de elementos finitos, caracterização do material e definição de condições de contorno entre outros itens.

O que foi apresentado neste trabalho foi uma proposta inovadora cujo propósito se baseou na restauração 2D para se obter uma mapeamento de informações tridimensionais a custo relativamente baixo e computacionalmente eficiente.

\section{Propostas de trabalhos futuros}






  \arial
  \bibliography{tiny}
  \normalfont
  %\input{appendix}

\end{document}

%% -*- coding: utf-8; -*-

\documentclass[
  master
  brazilian
]{ThesisPUC}


%%%
%%% Additional Packages
%%%

  \usepackage[brazilian]{babel}      %% in ThesisPUC.cls
  %% \usepackage[utf8]{inputenc}        %% .
  %% \usepackage[T1]{fontenc}           %% .
  %% \usepackage{lmodern}               %% .
  %% \usepackage[pdftex]{graphicx}	%% .

  \usepackage{tabularx}
  \usepackage{multirow}
  \usepackage{multicol}
  \usepackage{colortbl}
  \usepackage[%
    dvipsnames,
    svgnames,
    x11names,
    fixpdftex
  ]{xcolor}
  \usepackage{numprint}
  \usepackage{textcomp}
  \usepackage{booktabs}
  \usepackage{amsmath}
  \usepackage{enumitem}
  \usepackage{amssymb}
  \usepackage{textcomp}
% \usepackage{etoolbox}
  \usepackage{float}
  \usepackage[bottom]{footmisc}

%% numprint 
\npthousandsep{.}
\npdecimalsign{,}

%% ThesisPUC option
%\tablesmode{figtab} %% [nada, fig, tab ou figtab]
%\abreviationsmode{none} %% [none ou use] %% Default is [use]


%%%
%%% Counters
%%%

%% uncomment and change for other depth values
%% \setcounter{tocdepth}{3}
%% \setcounter{lofdepth}{3}
%% \setcounter{lotdepth}{3}
%% \setcounter{secnumdepth}{3}


%%%
%%% New commands and other global definitions
%%%

\input{defs}

%%%
%%% Misc.
%%%

\usecolour{true}

%%%
%%% Titulos
%%%

\author{Vinicius da Silva Costa Almada}
\authorR{Almada, Vinicius da Silva Costa}

\advisor{Luiz Fernando Martha}{Prof.}
\advisorR{Martha, Luiz Fernando}
% If the advisor's department is different from author's department, uncomment the next line and type the correct name and acronym of advisor's institution.
%\advisorInst{institution name}{acronym}

\coadvisor{André Luís Müller}{Dr.}
\coadvisorR{Müller, André Luís}
\coadvisorInst{Instituto Tecgraf/PUC-Rio}{Tecgraf/PUC-Rio}

%% \title{Desenvolvimento de um sistema de microscopia digital para
%%  classificação automática de tipos de hematita em minério de ferro}

\title{Mapeamento de Superfícies e Volume Baseado em Restauração de Seções Geológicas}

\titleuk{Surface and Volume Mapping Based on Restoration of Geological Sections}

%% \subtitulo{Aqui vai o subtitulo caso precise}

\day{30}
\month{Junho}
\year{2021}

\city{Rio de Janeiro}
\CDD{XYZ.AB}
\program{Engenharia Civil}
\school{Centro Técnico Científico}
\university{Pontifícia Universidade Católica do Rio de Janeiro}
\uni{PUC-Rio}

%%%
%%% Jury
%%%

\jury{%
  \jurymember{Banca Um}{Prof.}
    {Universidade Federal de Minas Gerais}{UFMG}
  \jurymember{Banca Dois}{Prof.}
    {Universidade Federal de Ouro Preto}{UFOP}
  \jurymember{Banca Três}{Dr.}
    {Centro de Tecnologia Mineral}{CETEM/MCTI}
  \jurymember{Banca Quatro}{Dr.}
    {Departamento de Engenharia Química e de Materiais}{PUC-Rio}
}

%%%
%%% Resume
%%%

\resume{%
  Bacharel em Engenharia Civil pelo Instituto Federal de Educação, Ciência e Tecnologia do Maranhão (IFMA), formou-se em 2018. Foi bolsista de programas de Iniciação Científica PIBITI – IFMA, com projeto de desenvolvimento de software para Mecânica dos Solos e um segundo na área de Dinâmica das Estruturas. Este último foi base para seu Trabalho de Conclusão de Curso, cujo objetivo foi desenvolver um software para análise dinâmica de estruturas sujeitas à carregamento sísmico. Desde o fim de 2019 atua no Instituto Tecgraf como bolsista no Grupo de Modelagem Geológica de Sistemas Petrolíferos}

%%%
%%% Acknowledgment (REMINDER TO SCHOLARSHIP STUDENTS. Do not forget to thank the agencies that supported your work.)
%%%

\acknowledgment{%
  \noindent Primeiro parágrafo de agradecimento ...
  \bigskip

  \noindent Segundo parágrafo de agradecimento ...
}

%%%
%%% Catalog prekeywords
%%%

\catalogprekeywords{%
  \catalogprekey{Engenharia Civil e Ambiental}%
  \catalogprekey{Engenharia}%
}

%%%
%%% Keywords
%%%

\keywords{%
  \key{Geologia Estrutural}
  \key{Mapeamento de Superfícies Geológicas}
  \key{Mapeamento de Volume Geológico}
}

\keywordsuk{%
  \key{Structural Geology}%
  \key{Mapping of Geological Sections}%
  \key{Mapping of Geological Surfaces}%
}

%%%
%%% Abstract
%%%

\abstract{%
  \textit{Ainda é o resumo do pré-projeto}

  Esse projeto visa desenvolver, dentro do Sistema Recon MS, um fluxo de trabalho que envolve as áreas de geologia estrutural, estratigrafia e geologia de reservatórios. Esse fluxo inicia com a restauração de modelos geológicos (seções e superfícies). Dentro do Sistema Recon busca-se aprimorar as ferramentas em seu ambiente de visualização 3D, denominado multi-seções, ou MS. Através de duas formas diferentes serão desenvolvidas ferramentas que irão prover as geometrias para o ambiente de simulação estratigráfica (para cada tempo geológico): 1) mapeamento de litologias e outras propriedades durante a restauração de seções, que por sua vez gerarão os paleo-relevos; 2) restauração de superfícies em si, que elimina a necessidade do mapeamento do item 1. A importância de ambas as estratégias consiste no fato de que nem sempre é possível obter bons resultados com restauração de superfícies, já que os mecanismos de restauração de seções desenvolvidos são muito mais geológicos. Para o mapeamento é necessária a criação de uma estrutura de dados que represente a malha de superfícies com possibilidade de armazenar informações, como propriedades e ligações entre as entidades das superfícies com as seções. Atualmente o Sistema Recon trabalha de forma totalmente desacoplado entre a estruturas de dados que representa as seções geológicas e a estrutura que representa as superfícies geológicas (horizontes, falhas e tipo do sal). Já para a segunda abordagem, ou seja, a restauração de superfícies, prevê-se o desenvolvimento de um módulo computacional de otimização dedicada a eliminar os buracos das superfícies que estão sendo restauradas incluindo restrições de deslocamento, sobre as falhas geológicas, das sub-superfícies abaixo da superfície restaurada.
}

\abstractuk{%
  This project aims to develop, within the Recon MS System, a workflow that involves the areas of structural geology, stratigraphy and reservoir geology. This flow begins with the restoration of geological models (sections and surfaces). The Recon System seeks to improve the tools in its 3D visualization environment, called multi-sections, or MS. Through two different forms, tools will be developed that will provide the geometries for the stratigraphic simulation environment (for each geological time): 1) mapping of lithologies and other properties during the restoration of sections, which in turn will generate paleo-reliefs; 2) surface restoration itself, which eliminates the need for mapping item 1. The importance of both strategies is the fact that it is not always possible to obtain good results with surface restoration, since the section restoration mechanisms developed are much more geological. For mapping, it is necessary to create a data structure that represents the surface mesh with the possibility of storing information, such as properties and links between the entities of the surfaces and the sections. Currently the Recon System works in a totally decoupled way between the data structures that represent the geological sections and the structure that represents the geological surfaces (horizons, faults and salt type). For the second approach, that is, the restoration of surfaces, it is foreseen the development of a computational optimization module dedicated to eliminate the holes of the surfaces that are being restored including displacement restrictions, on the geological faults, of the sub-surfaces below the restored surface.}

%%%
%%% Dedication
%%%

\dedication{%
  Dedicado lorem ipsum
}

%%%
%%% Epigraph
%%%

\epigraph{%
  My beautifull epigraph
}
\epigraphauthor{Wassily Kandinsky}
\epigraphbook{Regards sur le passé}

%%%
%%% Hyphenation
%%%

\hyphenation{PON-TI-FÍ-CIA}

%%%
%%%
%%% Quotes command
\newcommand{\quotes}[1]{``#1''}

%%%
%%% 
%%%

\begin{document}

  % -*- coding: utf-8; -*-

\chapter{Introdução}

This is the first chapter...



  % -*- coding: utf-8; -*-

\chapter{Processo de Restauração de Seções Geológicas}

\section{Sistema Recon MS}

\subsection{Introdução}

O ambiente no qual este trabalho é desenvolvido é o \textit{Sistema Recon MS}\cite{ReconTecgraf}, um software amplamente usado dentro da indústria de óleo e gás pela Petrobras e capaz de auxiliar na restauração de modelos geológicas. Conta com editor gráfico, estruturas de dados topológicos, algoritmos de transformações geológicas, gráficos de pós-processamentos entre outros recursos.

O Sistema Recon vem sendo desenvolvido a partir de um convênio entre o Instituto Tecgraf/PUC-Rio e a Petrobras desde 1991. Atualmente sua equipe responsável é formada pelo Grupo de Modelagem Digital em Geociências do Tecgraf. Uma imagem (Figura~\ref{fig-recon}) da tela inicial do programa é mostrada abaixo.

\begin{figure} [H]
  \begin{center}
    \includegraphics[width=\textwidth]{images/fig-recon}
    \caption{Captura de tela do Sistema Recon MS\cite{Recon}.}\label{fig-recon}
  \end{center}
\end{figure}

De acordo com Fossen\cite{Fossen}, restauração de seções geológicas pode ser entendida como uma manipulação da seção a fim de realizar a reconstituição dela ao seu estado anterior às deformações ocorridas ao longo do tempo, em outras palavras busca-se realizar uma retrodeformação na seção e utilizá-la na interpretação estrutural de uma região de interesse.

Neste capítulo são apresentadas as principais características do Sistema Recon MS para o objeto deste trabalho a fim de prover uma contextualização para o que é exibido nos demais capítulos. As próximas subseções tratam da descrição dos componentes principais e recursos básicos disponibilizados pelo sistema no processo de restauração de seções geológicas e também de visualização tridimensional do modelo. 

\subsection{Subdivisão Planar} % Falar do HED e da TopS

Uma seção geológica pode ser representada digitalmente por uma subdivisão planar uma vez que ela pode ser vista como um conjunto de polígonos que dividem o domínio da seção. Estes polígonos podem sofrer deformações e deslocamentos oriundos das transformações geológicos que a seção pode sofrer durante o balanceamento. Há ainda informações de adjacências entres essas porções que também precisam ser consideradas em um contexto computacional da seção geológica.

Na Figura~\ref{fig-subdivisao-planar} é possível perceber, por exemplo, que as camadas A, B e C possuem 3 blocos separados por falhas. Cada bloco é uma região fechada delimitada por um conjunto de segmentos. Deve-se observar ainda que essas regiões possuem atributos geológicos como idade, litologia, porosidade, etc.

\begin{figure} [h]
  \begin{center}
    \includegraphics[width=400pt]{images/fig-subdivisao-planar}
    \caption{Seção geológica como uma subdivisão planar.\cite{Ferraz}}\label{fig-subdivisao-planar}
  \end{center}
\end{figure}

Segundo Berg\cite{Berg}, uma subdivisão planar pode ser definida como uma subdivisão do plano através do uso de \textit{arestas}, \textit{vértices} e \textit{faces}. Essas são as entidades topológicas presentes em uma subdivisão planar, a face é como a região descrita anteriormente, delimitada por arestas (segmentos de curva); os vértices são os limites das arestas, sendo um para cada extremidade (podendo ser o mesmo vértice no início e no final da aresta).

A subdivisão planar precisa atender a alguns requisitos em relação às entidades topológicas: não deve haver vértices coincidentes; arestas só podem se cruzar em um vértice e faces também só se cruzam ou em um vértice, ou em uma aresta. Em outras palavras, não deve existir sobreposição de elementos topológicos. 

No entanto, há ainda um último componente topológico: o \textit{loop} ou \textit{laço} que é, de forma sucinta, um suconjunto conexo e ordenado de arestas. Com essa definição, a \textit{face} pode ser interpretada como uma união de laços, um deles sendo externo (delimitando a fronteira externa da face) e zero ou mais internos.

Em suma, a subdivisão planar tem os seguintes elementos topológicos:
\renewcommand{\labelitemi}{•}
\begin{itemize}
  \item \textbf{Vértice}: representa um ponto único dentro do plano.
  \item \textbf{Aresta}: segmento de curva com vértices como limites.
  \item \textbf{Laço} (loop): suconjunto conexo e ordenado de arestas.
  \item \textbf{Face}: região delimitada por um ou mais laços.
\end{itemize}

\subsection{Modelagem da Subdivisão Planar}

Para modelar a subdivisão planar dentro do Recon é utilizada a biblioteca computacional \textbf{HED} desenvolvida pelo Instituto Tecgraf/PUC-Rio e é a implementação de uma estrutura de dados topológicos baseada em arestas, a \textit{Half-Edge}\cite{HED}, uma das razões para esta escolha são as relações fixas de adjacência que uma aresta apresenta em relação às outras componentes topológicas. Uma aresta sempre é delimitada por dois vértices (distintos ou não) e é adjacente à duas faces.

O HED introduz uma nova entidade que explora bem essa característica denominada \textit{half-edge} ou \textit{semiaresta} que é uma referência ao \quotes{uso} da aresta por uma face. Dessa forma, no HED, cada aresta é formada por duas semiarestas. Cada semiaresta guarda uma referência para uma face e também para um vértice de origem. Isto dá uma orientação para a semiaresta que é usada para indicar o sentido positivo do loop das faces, por exemplo.

A estrutura HED tem um aspecto hierárquico de listas duplamente encadeada de elementos topológicos. No nível mais alto está a subdivisão planar, denominada como \textit{HedSolid}, então vêm \textit{HedFace}, \textit{HedLoop}, \textit{HedHalfEdge} e \textit{HedVtx} no nível mais baixo. A representação da aresta, \textit{HedEdge} encontra-se no mesmo nível da HedHalfEdge.

Uma propriedade importante em estruturas topológicas são as relações de adjacências entre suas componentes, a HED não provê de forma direta todas as relações, contudo é possível chegar às demais com uso de indireções. Por exemplo, partindo de uma aresta, como chegar às faces vizinhas? Basta ir às semiarestas da aresta, cada semiaresta possui referência para uma face.

Apresentado o HED e seus elementos, a associação com as entidades geológicas é intuitiva. Uma camada geológica é representada por uma face; as linhas de horizonte, falha ou sal têm como correspondente as arestas, por último, cada conjunto contínuo de faces é associado a um sólido\footnote{Os sólidos representam uma subdivisão planar e em alguns casos, a seção pode apresentar partes inteiramente descontínuas onde cada uma é um sólido diferente. Para casos onde é necessário sobreposição de partes, só é possível com a existência de mais de um sólido.}.

Destaca-se que a ideia de representar a seção geológica como uma subdivisão planar, ou uma estrutura HED, visa facilitar a criação e manipulação computacional da seção durante o processo de restauração. Todavia, a representação completa precisa levar em consideração também os atributos geológicos. Mais detalhes sobre a estrutura de dados HED podem ser encontrados em Mäntÿla~\cite{HED}, Arruda~\cite{Arruda} e Botsch \textit{et al.}~\cite{Botsch}.

\subsection{Atributos Geológicos}

Como já dito, os blocos que formam a seção geológica possuem propriedades próprias e precisam também estarem salvas na estrutura de dados topológica.

Cada entidade do HED possui um campo reservado para um tipo genérico de informações e neste espaço são organizados os atributos geológicos da seção. Estes atributos são representados em estruturas chamadas \textit{GeoSolid}, \textit{GeoFace}, \textit{GeoEdge} e \textit{GeoVtx}. Pela nomenclatura, é fácil observar a relação com o HED. As principais informações organizadas nessas estruturas são:

\renewcommand{\labelitemi}{•}
\begin{itemize}
  \item \textbf{GeoSolid}: o sólido por ser a estrutura de mais alto nível, é quem vai guardar referência à seção e ao cenário ao qual pertence dentro da restauração.
  \item \textbf{GeoFace}: é a estrutura que precisa armazenar dados do material geológico que a compõe (como idade, tipo, características físicas, etc.) e malha de triângulos que pode ser manipulada pelas transformações.
  \item \textbf{GeoEdge}: estrutura que guarda o tipo de linha (de horizonte, falha, topo de sal, etc.) e a subdivisão geométrica que forma a linha. 
  \item \textbf{GeoVtx}: é a única que armazena apenas o identificador universal.
\end{itemize}

Aliás, todas as estruturas de atributos geológicos possuem um campo para salvar este identificador que possui o formato \textit{UUID} --- \textit{universally unique identifier}~\cite{UUID} ou identificador único universal que é usado, por exemplo, na associação dos elementos geológicos com a malha triangular das faces, que será apresentada adiante.

\subsection{Seções Geológicas} % Falar da árvore de cenários

O principal recurso do Sistema Recon é seu conjunto de ferramentas para manipular uma seção geológica, desde a digitalização das informações que a definem geometricamente, da caracterização dos materiais e propriedades, da criação de dispositivos de controle e monitoramento da restauração até o kit de transformações que irão deformar a seção.

\subsubsection{Criação de uma seção geológica}

Para criar uma seção geológica no Sistema Recon pode-se recorrer ao editor gráfico para desenhar linhas e atribuir propriedades manualmente conforme seu tipo (se for horizonte, falha, limites da seção, etc.) ou em modelos que apresentem superfícies tridimensionais, como na Figura~\ref{fig-recon-1}, as seções podem ser criadas pela interseção de um plano vertical segundo uma direção dada pelo usuário, essa ação é chamada de \textit{fatiamento} do modelo.

\begin{figure} [H]
  \begin{center}
    \includegraphics[width=\textwidth]{images/fig-recon-1}
    \caption{Sistema Recon exibindo um modelo com superfícies tridimensionais e uma seção em destaque.}\label{fig-recon-1}
  \end{center}
\end{figure}

\subsubsection{Malhas da seção geológica}

A seção geológica é representada como uma subdivisão planar, como já citado, e é utilizada a biblioteca HED na implementação dessa subdivisão. Na Figura~\ref{fig-recon-2} pode-se observar uma seção geológica e alguns elementos, como as linhas (\textit{HedEdges}) onde a sua cor representa o atributo de tipo e as faces (\textit{HedFaces}) que são, em termos simples, regiões fechadas por linhas. Neste exemplo, todas as faces pertencem à mesma camada geológica.

\begin{figure} [h]
  \begin{center}
    \includegraphics[width=\textwidth]{images/fig-recon-2}
    \caption{Seção geológica com destaque para os elementos de linhas e faces.}\label{fig-recon-2}
  \end{center}
\end{figure}

As faces têm um atributo muito importante para o trabalho de restauração, que são as malhas de triângulos. Cada face possui uma malha independente das outras. No Sistema Recon, essa malha é armazenada numa estrutura de dados topológicos chamada \textit{TopS} que trata-se de uma biblioteca computacional voltada para representação de malhas de elementos finitos.\cite{Tops} A Figura~\ref{fig-recon-3} exibe a mesma seção, mas com adição das malhas das faces.

\begin{figure} [H]
  \begin{center}
    \includegraphics[width=350pt]{images/fig-recon-3}
    \caption{Malhas das faces de uma seção geológica no Sistema Recon.}\label{fig-recon-3}
  \end{center}
\end{figure}

A importância das malhas dentro da restauração de seções no Sistema Recon se dá por conta das transformações geológicas que possuem como requisito de entrada uma malha. Um pouco mais de detalhes sobre as transformações será apresentado adiante.

A estrutura \textit{TopS} permite armazenar certos atributos em seus elementos topológicos. Em especial nos vértices da malha, no Sistema Recon, é armazenado o \textit{UUID} do atributo geológico da entidade topológica do HED sobre a qual aquele vértice está, em outras palavras, se o vértice da malha está no interior da face, ele guarda o \textit{UUID} da \textit{GeoFace} dessa face, o mesmo para caso esteja sobre uma aresta (\textit{GeoEdge}) ou vértice (\textit{GeoVertex}). A Figura~\ref{fig-recon-4} mostra um exemplo da forma como esses dados são obtidos.

\begin{figure} [H]
  \begin{center}
    \includegraphics[width=\textwidth]{images/fig-recon-4}
    \caption{Trecho de uma malha de face com os tipos de atributo geológicos que estão sob os vértices da malha.}\label{fig-recon-4}
  \end{center}
\end{figure}

Essa relação permite identificar, a partir de um vértice da malha, sobre qual entidade geológica está este vértice, este recurso será usado no próximo capítulo.

\subsubsection{Transformações}

As transformações geológicas são ferramentas que buscam reverter (ou simular) as movimentações e deformações ocorridas ao longo do tempo.\cite{Santi} As transformações são aplicadas diretamente às malhas, no entanto, para que isso aconteça, é necessário antes a definição de \textit{Módulos} na seção. 

Módulos são agrupamentos de faces ou blocos da seção, têm o intuito de reunir aquelas partes que devem ter recebido as mesmas deformações, podendo ser, inclusive partes de camadas diferentes.

Com o módulo definido, consegue-se aplicar uma transformação. Esta irá ser empregada sobre a malha de cada uma das faces que compõem aquele módulo, deformando esta malha e, por conseguinte, alterando a geometria da seção.

A Figura~\ref{fig-recon-5} mostra a aba \quotes{Transformações} do Sistema Recon onde é possível ver os grupos de transformações e também as opções disponíveis. Mais detalhes sobre cada uma delas podem ser consultados no manual do usuário do programa\cite{Recon}.

\begin{figure} [H]
  \begin{center}
    \includegraphics[width=\textwidth]{images/fig-recon-5}
    \caption{Aba \quotes{Transformações} do Sistema Recon.}\label{fig-recon-5}
  \end{center}
\end{figure}

\subsubsection{Árvore de cenários}

A restauração de seções é um processo linear no sentido de que cada novo passo depende de como estava o anterior. Qualquer mudança num passo desses acarreta em um resultado diferente ao final. Além do mais, balanceamento de seções não é uma atividade de resposta única, o objetivo é obter uma interpretação geológica restaurável, ou seja, seus componentes estruturais devem poder ser restauradas~\cite{Fossen}. 

Diante disto, o Sistema Recon disponibiliza em sua interface de manipulação das seções um componente capaz de registrar o histórico de etapas no processo de restauração, mais que isso, ao usuário é dada a possibilidade de voltar em algum ponto e criar uma nova linha de estudo dentro desse processo, ou ainda apagar uma sequência de etapas que ele julga estar incorreta.

Isso tudo é possível graças à árvore de cenários. Um cenário é a representação de um estado de restauração de uma seção. Por exemplo, se de um passo a outro da restauração ocorre uma transformação, o estado anterior pode ser registrado em um cenário e o posterior em um outro. De cada cenário pode-se criar diversos outros como se fossem diferentes linhas do tempo, ou diferentes interpretações daquele passo de restauração.

Árvores são um tipo especial de estrutura de dados não-linear e neste casos de uso é definida como tendo uma raiz ou nó inicial que aponta para um ou mais nós. Estes, igualmente, podem apontar para diferentes nós numa escala hierárquica. A Figura~\ref{fig-recon-6} apresenta um exemplo de árvore de cenários tirada do Sistema Recon. Nesta imagem, cada quadrinho representa a seção num dado estado e como identificação, cada cenário também possui um número.

\begin{figure} [H]
  \begin{center}
    \includegraphics[width=160pt]{images/fig-recon-6}
    \caption{Exemplo de árvore de cenários de uma seção do Sistema Recon.}\label{fig-recon-6}
  \end{center}
\end{figure}

O primeiro cenário tem a seção em sua versão inicial e a cada nova manipulação da mesma, pode-se criar um novo cenário e assim ter este histórico. Essa maneira de organizar uma restauração é útil não só no contexto de uma seção isolada, mas principalmente quando se trabalha em modelos de multisseções que irão sofrer os mesmos processos de restauração, mas de maneiras diferentes. Com um registro do quê e quando ocorreu uma dada transformação em diferentes seções é possível ter uma visão mais geral do modelo em uma sequência cronológica.

\subsection{Ambiente Multisseções}

Apesar dos principais recursos do Sistema Recon atuarem diretamente com seção geológica, isso não significa dizer que só seja possível manipular modelos com uma única seção geológica. Uma das grandes mudanças ocorridas no programa foi a criação de ferramentas para se tratar de modelos com múltiplas seções, ou modelos multisseção~\cite{Felipe},~\cite{Garcia}.

O ambiente multisseção (MS) do Sistema Recon trata-se de um visualizador 3D onde podem ser vistas as superfícies geológicas e também as seções em um contexto global do modelo.

Como sistema de referências, o ambiente MS usa coordenadas UTM (Universal Transversa de Mercador)~\cite{IBGE} para localizar seus objetos. Neste sistema, cada ponto é representado por um par $(N, E)$ onde $N$ é a coordenada norte-sul em metros e $E$, a coordenada leste-oeste.

A Figura~\ref{fig-recon-7} exibe o Sistema Recon no ambiente MS, onde é possível notar o (1) visualizador tridimensional com superfícies e seções geológicas, (2) a lista de \textit{EtapasMS} que será apresentada a seguir juntamente da (3) lista de cenários da etapa.

\begin{figure} [H]
  \begin{center}
    \includegraphics[width=\textwidth]{images/fig-recon-7}
    \caption{Ambiente Multisseção do Sistema Recon.}\label{fig-recon-7}
  \end{center}
\end{figure}

\subsection{Etapas de restauração}

Como brevemente apresentado, o ambiente MS permite ter uma olhar mais global do modelo e de todos os componentes que o formam. Neste contexto é então preciso organizar as seções de forma que haja o máximo de coerência do ponto de vista geral durante a restauração do modelo. Podem existir seções que compartilham uma mesma falha ou um mesmo evento tectônico, por exemplo.

Como já visto, cada seção conta com um registro de cada passo dado no andamento da restauração e é um recurso presente apenas localmente e independente. No entanto, seções relativamente próximas, ou que foram restauradas de maneira semelhante precisam sincronizar esse histórico para que haja uma ordem melhor do modelo sob um aspecto global.

Com essa finalidade, foi criado e implementado o conceito de \textit{EtapaMS}. Uma \textit{EtapaMS} trata-se de um conjunto de cenários de seções diferentes mas que, de certa forma, representam o mesmo marco geológico, como a restauração de uma falha ou descompactação. Cada \textit{EtapaMS} pode ter apenas 1 cenário por seção dentro de sua estrutura, isso permite ter um histórico do modelo multisseção análogo à árvore de cenário da seção individualmente.

No Sistema Recon, as \textit{EtapasMS} são dispostas em lista no ambiente multisseção. Ao selecionar um item dessa lista, logo abaixo é exibido o conjunto de cenários (por seção) que compõem aquela \textit{EtapaMS}, como bem mostra a Figura~\ref{fig-recon-7}.

Uma forma de uso das \textit{EtapasMS} para a restauração de modelos geológicos é organizar os estados de seções diferentes que respondam ao mesmo evento ou marco geológico. Caso haja uma falha X que atravessa 3 seções e em todas elas essa falha é restaurada, pega-se o cenário de cada seção onde isso ocorre e cria-se uma \textit{EtapaMS} correspondente a este marco. Esta organização da restauração é parte importante para o mapeamento de superfícies e volume.



  %% -*- coding: utf-8; -*-

\chapter{Sistema Recon MS}

\section{Introdução}

O ambiente no qual este trabalho é desenvolvido é o \textit{Sistema Recon MS}, um sistema computacional capaz de auxiliar no balanceamento de seções geológicas. Conta com editor gráfico, estruturas de dados topológicos, algoritmos de transformações geométricas, relatórios de pós-processamentos entre outros recursos.

O Recon MS é desenvolvido a partir de um convênio entre o Instituto Tecgraf/PUC-Rio e a Petrobras desde 1991.

Neste capítulo são apresentadas as principais características do Sistema Recon MS para o objeto deste trabalho a fim de prover uma contextualização para o que é exibido nos demais capítulos. As primeiras seções tratam da descrição dos recursos básicos usados pelo sistema no processo de restauração de seções geológicas. Ao fim, é mostrado a definição da linhas de mapeamento, bem como o tipo de mapeamento desenvolvido com base em tais linhas.

\section{Subdivisão Planar} % Falar do HED e da TopS

Uma seção geológica pode ser vista como um conjunto de polígonos onde cada um representa uma porção limitada com suas próprias características. Estes polígonos podem sofrer deformações e deslocamentos oriundos das transformações geométricas às quais a seção pode sofrer durante o balanceamento. Há ainda informações de adjacências entres essas porções que também precisam ser consideradas em uma representação digital da seção geológica.



\section{Atributos Geológicos}

\section{Seções Geológicas} % Falar da árvore de cenários

\section{Módulos}

\section{Transformações}

\section{Malhas e Resultados}

This is the first chapter

\section{Linhas de Mapeamento}

Um dos recursos presentes no Sistema Recon é a \textbf{linha de mapeamento}, cujo objetivo é auxiliar na interpretação dos resultados gerados na restauração do modelo. Essa linha armazena referências a pontos topológicos da malha da seção. Com isso, é possível ter uma linha que acompanha a movimentação da malha de um cenário a outro.

As linhas de mapeamento (Figura~\ref{fig-linemap}) permitem realizar um mapeamento geométrico ao longo de uma restauração tomando como base uma linha-guia poligonal definida em um dado cenário. Essa linha pode ser criada em qualquer cenário, mesmo em seções já restauradas.

\begin{figure} [h]
  \begin{center}
    \includegraphics[width=400pt]{images/fig-linhas-de-mapeamento-ed}
    \caption{Linhas de mapeamento em uma seção.}\label{fig-linemap}
  \end{center}
\end{figure}

Cada face de uma seção tem como atributo uma malha triangulada, e as linhas de mapeamento são definidas no sistema de coordenadas local da malha de cada uma das faces. Além disso, é possível que uma linha de mapeamento cruze diversas malhas, por isso, a linha de mapeamento é definida como um conjunto de "partes" de linha de mapeamento, sendo cada parte pertencente a um trecho contínuo em uma mesma face. O processo de criação do mapeamento da linha é feito para cada parte. Na Figura~\ref{fig-linemap-malhas} é possível ver uma linha de mapeamento cortando algumas malhas diferentes.

\begin{figure} [h]
  \begin{center}
    \includegraphics[width=350pt]{images/fig-linhas-de-mapeamento-malhas}
    \caption{Linhas de mapeamento cortando múltiplas faces.}\label{fig-linemap-malhas}
  \end{center}
\end{figure}

A primeira etapa desse mapeamento é a criação da linha-guia, a partir disso é feita a separação nas partes a serem processadas. É realizado um mapeamento com informações topológicas da interseção da parte com a malha. Essa ação consiste em fazer uma relação entre um ponto da parte da linha-guia e um ponto em uma entidade topológica da malha.

Por exemplo, na Figura~\ref{fig-linemap-parts} estão evidenciadas as partes que formam a linha de mapeamento. Cada uma dessas partes é representada pela entidade chamada \textit{LineMapPart}.

\begin{figure} [h]
  \begin{center}
    \includegraphics[width=350pt]{images/fig-lm-parts}
    \caption{Partes de uma linha de mapeamento}\label{fig-linemap-parts}
  \end{center}
\end{figure}

Cada parte é processada individualmente e, como já citado, a linha de mapeamento final é um conjunto dessas partes.

Como a malha possui três entidades básicas, o ponto da linha pode ser mapeado para um nó topológico, para um ponto interno de uma aresta (lado de triângulo) de uma malha ou ponto interior a um elemento (triângulo da malha). Em cada um desses casos, a informação topológica relacionada é guardada:

\renewcommand{\labelitemi}{•}
\begin{itemize}
  \item Nó: guarda o indentificador do nó
  \item Aresta: guarda o identificador da aresta e a coordenadas paramétricas do ponto em que cruza a aresta.
  \item Elemento: guarda o identificador do elemento e as coordenadas baricêntricas do ponto no interior do elemento.
\end{itemize}

A Figura~\ref{fig-lm-topo} mostra a idenficação dos pontos da linha de mapeamento e a Tabela~\ref{tab-lm-topo} exibe quais informações topológicas são salvas de cada ponto.

\begin{figure} [hbt!]
  \begin{center}
    \includegraphics[width=260pt]{images/fig-lm-topo}
    \caption{Informações topológicas da malha mapeadas para a linha de mapeamento.}\label{fig-lm-topo}
  \end{center}
\end{figure}

% -*- coding: utf-8; -*-

\begin{table} [hbt!]
 \begin{center}
	 \caption{Informações topológicas salvas na linha de mapeamento.\label{tab-lm-topo}}
	~\\[-2mm]
	 \begin{tabularx}
		 {\textwidth}
		 {cp{2.0cm} lp{3.0cm} lp{10.0cm}}

		 \textbf{Ponto}
		 & \textbf{Tipo}
		 & \textbf{Informação armazenada} \\ \toprule

		 %~\\[-1mm]
		 A
		 & Elemento
		 & id=30, coordenadas baricêntricas=(0,33; 0,33; 0,33) \\ \midrule

		 %~\\[-1mm]
		 B
		 & Nó   
		 & id=431 \\ \midrule

		 %~\\[-1mm]
		 C
		 & Aresta
		 & id=130, coordenada paramétrica=0,45 \\ \midrule

		 %~\\[-1mm]
		 D
		 & Aresta
		 & id=145, coordenada paramétrica=0,55 \\ \midrule

	 \end{tabularx}
 \end{center}
\end{table}


Após esse processo de mapeamento topológico da linha propriamente dita, é possível calcular a geometria da linha em diferentes cenários que usam a mesma malha (já que a topologia é mantida), bastando apenas verificar se a malha se manteve, isto é, não foi refeita, apagada ou editada.

Em casos de edição, todos os atributos associados à malha são interpolados oara a nova versão da malha, incluem-se nisso as partes de linha de mapeamento que recebem uma nova versão se a malha original teve sua topologia alterada.

A vantagem deste tipo de mapeamento é ser baseado em malha, já que todas as transformações geológicas que ocorrem no processo de restauração, tem como objetivo deformar a malha.

\section{Derivações das Linhas de Mapeamento}

As linhas de mapeamento têm também casos de usos mais especializados, como na criação e representação de poços. Poços são criados semelhantemente às linhas de mapeamento ou por importação de modelos com poços em 3D. Têm característica de serem linhas mais verticalizadas e possuem uma finalidade mais limitada. Nos casos de poços 3D, a linha correspondente ao poço é apenas uma projeção do objeto tridimensional.

Há o uso nas chamadas linhas de interseção (\textit{CrossLine}) que servem para identificar e mapear as linhas de cruzamento entre seções no espaço tridimensional do multi-seções, com isso é possível ter uma noção do que ocorre com seções transversais mesmo estando no domínio bidimensional da restauração.

Por fim, as linhas de mapeamento são a base para a \textit{linhas de mapeamento do modelo} ou \textit{LMModel}, cujo objetivo é servir como um mapeamento das linhas de entidades geológicas (horizonte, falha e topo de sal) ao longo da restauração do modelo. Dessa forma, é possível ter um acompanhamento das entidades geológicas na seção, baseado no contorno da malha das faces que são mais bem discretizadas que as arestas originais da subdivisão planar, além de poder verificar como se deu a movimentação de cada ponto de horizonte ao longo da restauração, por exemplo.

Pelo objetivo proposto, as LMModels são linhas de mapeamento que tomam a geometria das entidades geológicas como entrada, então não há necessidade de criar uma linha-guia como é feita na linha de mapeamento original. Neste caso, há uma parte de LMModel para cada aresta, seja de horizonte, falha ou topo de sal.

Além disso, há o armazenamento de atributos importantes para a manipulação das LMModels, como idade dos horizontes, identificador da falha e até sobre a qual pedaço de superfície aquela parte de linha está associada.

Todas essas informações  geológicas atreladas ao mapeamento topológico das LMModels, quando em conjunto com as diversas seções geológicas de um modelo multi-seções, são o que fazem dela o principal dado para a realização de uma restauração e um mapeamento de informações a nível 3D, já que trazem todo o histórico de movimentação das camadas de um modelo geológico.

A maneira de trabalhar com LMModels é com a organização dela em estruturas de dados que formam subconjuntos divididos por etapa de resturação e idade (caso de linhas de horizonte). Com isso é obtido o conjunto de informações que representam a restauração das seções de uma forma mais simples e fácil de se trabalhar a nível tridimensional.



  % -*- coding: utf-8; -*-

\chapter{Mapeamento}

\section{Linhas de Mapeamento}

Um dos recursos presentes no Sistema Recon é a \textbf{linha de mapeamento}, cujo objetivo é auxiliar na interpretação dos resultados gerados na restauração do modelo. Essa linha armazena referências a pontos da malha da seção. Com isso, é possível ter uma linha que acompanha a movimentação da malha de um cenário a outro.

As linhas de mapeamento (Figura~\ref{fig-linemap}) permitem realizar um mapeamento geométrico ao longo de uma restauração tomando como base uma linha-guia poligonal definida pelo usuário. Essa linha pode ser criada em qualquer cenário, mesmo em seções já restauradas.

\begin{figure} [h]
  \begin{center}
    \includegraphics[width=400pt]{images/fig-linhas-de-mapeamento-ed}
    \caption{Linhas de mapeamento em uma seção.}\label{fig-linemap}
  \end{center}
\end{figure}

A Figura~\ref{fig-linemap-history} apresenta o resultado após uma transformação do tipo \textit{Move-Sobre-Falha} onde é possível observar, além da deformação da camada, a linha de mapeamento sofrendo a mesma movimentação. Este tipo de uso pode ser interpretado como se houvesse ali um falso horizonte para avaliar o quantidade de movimento na restauração do rejeito.

\begin{figure} [h]
  \begin{center}
    \includegraphics[width=420pt]{images/fig-linemap-history}
    \caption{Linhas de mapeamento em diferentes etapas}\label{fig-linemap-history}
  \end{center}
\end{figure}


Cada face de uma seção tem como atributo uma malha triangulada, e as linhas de mapeamento são definidas no sistema de coordenadas local da malha de cada uma das faces. Além disso, é possível que uma linha de mapeamento cruze diversas regiões da seção, por isso, a linha de mapeamento é definida como um conjunto de \quotes{partes de linha de mapeamento} ou \textit{LineMapPart}, sendo cada parte pertencente a um trecho contínuo em uma mesma face. O processo de criação deste mapeamento da linha é feito para cada parte individualmente, de forma que ao visualizar as partes tem-se a linha de mapeamento completa. Na Figura~\ref{fig-linemap-malhas} é possível ver uma linha de mapeamento cortando algumas regiões diferentes.

\begin{figure} [h]
  \begin{center}
    \includegraphics[width=250pt]{images/fig-linhas-de-mapeamento-malhas}
    \caption{Linhas de mapeamento cortando múltiplas faces.}\label{fig-linemap-malhas}
  \end{center}
\end{figure}

O primeiro passo na criação do mapeamento é a determinação da linha-guia pelo usuário, a partir disso é feita a separação nas partes a serem processadas de acordo com o número de faces interceptadas. É realizado um levantamento com informações topológicas da interseção da parte com a malha da face. Essa ação consiste em fazer uma relação entre um ponto da parte da linha-guia e um ponto em uma entidade topológica da região, seja ela um nó, aresta de elemento ou elemento triangular.

Na Figura~\ref{fig-linemap-parts} estão evidenciadas as partes que formam a linha de mapeamento. Cada uma dessas partes é representada pela entidade chamada \textit{LineMapPart}, termo apresentado anteriormente.

\begin{figure} [h]
  \begin{center}
    \includegraphics[width=250pt]{images/fig-lm-parts}
    \caption{Partes de uma linha de mapeamento}\label{fig-linemap-parts}
  \end{center}
\end{figure}

A malha de uma face, por sua vez, possui três entidades básicas, o ponto da linha pode ser mapeado para um nó topológico, para um ponto interno de uma aresta (lado de triângulo) da malha ou ponto interior a um elemento (triângulo da malha). Em cada um desses casos, a informação topológica relacionada é guardada, segundo a enumeração a seguir:

\renewcommand{\labelitemi}{•}
\begin{itemize}
  \item Nó: guarda o identificador do nó
  \item Aresta: guarda o identificador da aresta e a coordenada paramétrica do ponto de cruzamento.
  \item Elemento: guarda o identificador do elemento e as coordenadas baricêntricas do ponto no interior do elemento.
\end{itemize}

A Figura~\ref{fig-lm-topo} mostra a identificação dos pontos em uma \textit{LineMapPart} e a Tabela~\ref{tab-lm-topo} exibe quais informações topológicas são salvas de cada ponto.

\begin{figure} [hbt!]
  \begin{center}
    \includegraphics[width=260pt]{images/fig-lm-topo}
    \caption{Informações topológicas da malha mapeadas para a linha de mapeamento.}\label{fig-lm-topo}
  \end{center}
\end{figure}

% -*- coding: utf-8; -*-

\begin{table} [hbt!]
 \begin{center}
	 \caption{Informações topológicas salvas na linha de mapeamento.\label{tab-lm-topo}}
	~\\[-2mm]
	 \begin{tabularx}
		 {\textwidth}
		 {cp{2.0cm} lp{3.0cm} lp{10.0cm}}

		 \textbf{Ponto}
		 & \textbf{Tipo}
		 & \textbf{Informação armazenada} \\ \toprule

		 %~\\[-1mm]
		 A
		 & Elemento
		 & id=30, coordenadas baricêntricas=(0,33; 0,33; 0,33) \\ \midrule

		 %~\\[-1mm]
		 B
		 & Nó   
		 & id=431 \\ \midrule

		 %~\\[-1mm]
		 C
		 & Aresta
		 & id=130, coordenada paramétrica=0,45 \\ \midrule

		 %~\\[-1mm]
		 D
		 & Aresta
		 & id=145, coordenada paramétrica=0,55 \\ \midrule

	 \end{tabularx}
 \end{center}
\end{table}


Após esse processo de mapeamento topológico da linha propriamente dita, é possível calcular a geometria da linha em diferentes cenários que usam a mesma malha (já que a topologia é mantida), bastando apenas verificar se a malha se manteve, isto é, não foi refeita, apagada ou editada.

Em casos de edição, todos os atributos associados à malha são interpolados para a nova versão da malha, incluem-se nisso as partes de linha de mapeamento que recebem uma nova versão se a malha original teve sua topologia alterada.

A vantagem deste tipo de mapeamento é ser baseado em malha, já que em todas as transformações geológicas que ocorrem no processo de restauração só malhas das faces são deformadas.

\section{Derivações das Linhas de Mapeamento}

As linhas de mapeamento têm também casos de usos mais especializados, como na criação e representação de poços. Poços são criados semelhantemente às linhas de mapeamento ou por importação de modelos com poços em 3D. Possuem característica de serem linhas mais verticalizadas e possuem uma finalidade mais limitada. Nos casos de poços 3D, a linha correspondente ao poço é apenas uma projeção do objeto tridimensional no plano da seção.

Há o uso nas chamadas linhas de interseção (\textit{CrossLine}) que servem para identificar e mapear as linhas de cruzamento entre seções no espaço tridimensional do multi-seções, com isso é possível ter uma noção do que ocorre com seções transversais mesmo estando no domínio bidimensional da restauração.

Por fim, as linhas de mapeamento são a base para a \textit{linhas de mapeamento do modelo} ou \textit{LMModel}, cujo objetivo é servir como um mapeamento das linhas de entidades geológicas (horizonte, falha e topo de sal) ao longo da restauração do modelo. Dessa forma, é possível ter um acompanhamento das entidades geológicas na seção, baseado no contorno da malha das faces, além de poder verificar como se deu a movimentação de cada ponto de horizonte, falha ou topo de sal ao longo da restauração, por exemplo.

Pelo objetivo proposto, as \textit{LMModels} são linhas de mapeamento que tomam a geometria das entidades geológicas como entrada, então não há necessidade de criar uma linha-guia como é feita na linha de mapeamento original, a própria linha de horizonte ou falha é usada como linha-guia.

Além disso, há o armazenamento de atributos importantes para a manipulação das LMModels, como idade dos horizontes, identificador da falha e até sobre a qual pedaço de superfície aquela parte de linha está associada.

Todas essas informações  geológicas atreladas ao mapeamento topológico das LMModels, quando em conjunto com as diversas seções geológicas de um modelo multi-seções, são o que fazem dela o principal dado para a realização de um mapeamento de informações a nível 3D, já que trazem todo o histórico de movimentação das camadas de um modelo geológico.

A maneira de trabalhar com LMModels é com a organização dela em estruturas de dados que formam subconjuntos divididos por etapa de resturação e idade (caso de linhas de horizonte). Com isso é obtido o conjunto de informações que representam a restauração das seções de uma forma mais simples e fácil de se trabalhar a nível tridimensional.

\section{Mapeamento de superfícies}

\subsection{Metodologia}

\subsection{Preparação dos dados}

\subsection{Exemplos e resultados}

\section{Mapeamento do Volume}

\subsection{Metodologia}

\subsection{Preparação dos dados}

\subsection{Exemplos e resultados}



  % -*- coding: utf-8; -*-

\chapter{Conclusão}

\section{Resumo}

Este trabalho propôs um mapeamento do comportamento de superfícies e volume baseado na restauração de seções geológicas. Uma maneira de levantar informações a respeito do comportamento tridimensional a partir da movimentação tectônica gerada pelo balanceamento das seções geológicas.

Inicialmente foi apresentado uma breve caracterização do que se trata a restauração de seções geológicas com enfoque maior naquilo que o Sistema Recon MS, base de desenvolvimento, oferece para realizar tal atividade. Dentre os recursos fornecidos pode-se citar a estrutura de dados topológicos HED, geração de malhas em seções, transformações geométricas e por fim, as linhas de mapeamento.

Este mapeamento realizado nas seções foi base para a criação das \emph{LMModels} e com isso se obteve o mapeamento das entidades geológicas ao longo da restauração. Ao exportar as \emph{LMModels} para um ambiente multisseções, foi possível associar este mapeamento às superfícies tridimensionais. A organização e uso desses dados de \emph{LMModels} serviram como parâmetros para, através de um método numérico, realizar uma deformação às superfícies a fim de submetê-las à mesma movimentação das seções. O responsável por essa deformação vem do processo de minimização de uma energia de alta ordem, tri-harmônica, que resulta em uma superfície de mínima variação.

Em relação ao volume, idealizou-se que ele fosse discretizado em uma nuvem de pontos contida no domínio do modelo. Esses pontos devem sofrer uma movimentação segundo o deslocamento dos pontos das seções de uma \emph{EtapaMS} a outra. A fim de melhorar essa movimentação, podem ser usados os pontos das superfícies que foram deformadas no mapeamento de superfícies em conjunto dos pontos das seções. A nuvem de pontos é criada a partir de uma grade volumétrica (cada célula da grade com 8 pontos). Os pontos das seções e superfícies, que possuem posição final definida, são os que movimentam os pontos da grade. 

\section{Mapeamento x Restauração}

A restauração de seções geológicas é uma atividade já consagrada na interpretação estrutural e fornece resultados satisfatórios de maneira eficiente. No entanto, ao considerar apenas seções transversais em modelos geológicos, que são tridimensionais, pode, em algum momento, acarretar na dissociação do comportamento real ocorrido. Em razão disso, pode-se partir a estudos que também levem em conta características 3D. Por outro lado, soluções assim podem ser custosas e ineficientes.

Este trabalho realizou uma busca e tratamento de informações provenientes da restauração de seções e as aplicou para deformar as superfícies e movimentar uma nuvem de pontos. Esse processo foi chamado de mapeamento pois trata-se de um levantamento e processamento de dados com o objetivo de caracterizar uma entidade geológica, assim como é feito no mapeamento geológico~\cite{Geoscan}. Neste caso, o objetivo foi realizar uma definição do comportamento das superfícies geológicas e do volume com os dados da restauração de seções.

Uma restauração de superfícies geológicas seria um processo análogo à restauração de seções, no entanto, com suas premissas geológicas e métodos próprios. É possível que nem sejam usadas seções transversais para esse intuito. E se incluir o volume para uma restauração do modelo 3D diretamente, mais questões acerca da metodologia precisam ser feitas. 

Uma restauração 3D deve levar em conta ainda mais aspectos, como parâmetros geomecânicos além dos geométricos. Alguns autores~\cite{Santi_3dgeological, Massot, DURANDRIARD2010441} já apresentaram propostas para esta finalidade. Propostas estas que incluem uma malha volumétrica, uso de elementos finitos, caracterização do material e definição de condições de contorno entre outros itens.

O que foi apresentado neste trabalho foi uma proposta inovadora cujo propósito se baseou na restauração 2D para se obter uma mapeamento de informações tridimensionais a custo relativamente baixo e computacionalmente eficiente.

\section{Propostas de trabalhos futuros}






  \arial
  \bibliography{tiny}
  \normalfont
  \input{appendix}

\end{document}
